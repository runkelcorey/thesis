% This is the Reed College LaTeX thesis template. Most of the work
% for the document class was done by Sam Noble (SN), as well as this
% template. Later comments etc. by Ben Salzberg (BTS). Additional
% restructuring and APA support by Jess Youngberg (JY).
% Your comments and suggestions are more than welcome; please email
% them to cus@reed.edu
%
% See http://web.reed.edu/cis/help/latex.html for help. There are a
% great bunch of help pages there, with notes on
% getting started, bibtex, etc. Go there and read it if you're not
% already familiar with LaTeX.
%
% Any line that starts with a percent symbol is a comment.
% They won't show up in the document, and are useful for notes
% to yourself and explaining commands.
% Commenting also removes a line from the document;
% very handy for troubleshooting problems. -BTS

% As far as I know, this follows the requirements laid out in
% the 2002-2003 Senior Handbook. Ask a librarian to check the
% document before binding. -SN

%%
%% Preamble
%%
% \documentclass{<something>} must begin each LaTeX document
\documentclass[12pt,oneside]{psthesis}
% Packages are extensions to the basic LaTeX functions. Whatever you
% want to typeset, there is probably a package out there for it.
% Chemistry (chemtex), screenplays, you name it.
% Check out CTAN to see: http://www.ctan.org/
%%
\usepackage{graphicx,latexsym}
\usepackage{subfig}
\usepackage{amsmath}
\usepackage{amssymb,amsthm}
\usepackage{longtable,booktabs,setspace}
\usepackage{chemarr} %% Useful for one reaction arrow, useless if you're not a chem major
\usepackage[none]{hyphenat}
\usepackage[hyphens]{url}
% Added by CII
\usepackage[hidelinks]{hyperref}
\usepackage{breakurl}
\usepackage{lmodern}
\usepackage{float}
\usepackage{titlesec}
\floatplacement{figure}{H}
% End of CII addition
\usepackage{rotating}
\usepackage{epigraph}

% Next line commented out by CII
%%% \usepackage{natbib}
% Comment out the natbib line above and uncomment the following two lines to use the new
% biblatex-chicago style, for Chicago A. Also make some changes at the end where the
% bibliography is included.
%\usepackage{biblatex-chicago}
%\bibliography{thesis}


% Added by CII (Thanks, Hadley!)
% Use ref for internal links
\renewcommand{\hyperref}[2][???]{\autoref{#1}}
\def\chapterautorefname{Chapter}
\def\sectionautorefname{Section}
\def\subsectionautorefname{Subsection}
% End of CII addition

% Added by CII
\usepackage[font=singlespacing]{caption}
\captionsetup{margin=10pt}
% End of CII addition

% \usepackage{times} % other fonts are available like times, bookman, charter, palatino

% Syntax highlighting #22

% To pass between YAML and LaTeX the dollar signs are added by CII
\title{How Foreclosure Relief Swung America Right}
\author{Corey Runkel}
\computingid{\href{mailto:cnr3cg@virginia.edu}{\nolinkurl{cnr3cg@virginia.edu}}}
% The month and year that you submit your FINAL draft TO THE LIBRARY (May or December)
\date{20 April 2020}
\advisor{Herman M. Schwartz}
\institution{University of Virginia}
\degree{Bachelor of Arts with honors}
%If you have two advisors for some reason, you can use the following
% Uncommented out by CII
% End of CII addition

%%% Remember to use the correct department!
\department{Program in Political \& Social Thought}
% if you're writing a thesis in an interdisciplinary major,
% uncomment the line below and change the text as appropriate.
% check the Senior Handbook if unsure.
%\thedivisionof{The Established Interdisciplinary Committee for}
% if you want the approval page to say "Approved for the Committee",
% uncomment the next line
%\approvedforthe{Committee}

% Added by CII
%%% Copied from knitr
%% maxwidth is the original width if it's less than linewidth
%% otherwise use linewidth (to make sure the graphics do not exceed the margin)
\makeatletter
\def\maxwidth{ %
  \ifdim\Gin@nat@width>\linewidth
    \linewidth
  \else
    \Gin@nat@width
  \fi
}
\makeatother

\renewcommand{\contentsname}{Table of Contents}
% End of CII addition

\setlength{\parskip}{0pt}

% Added by CII

\providecommand{\tightlist}{%
  \setlength{\itemsep}{0pt}\setlength{\parskip}{0pt}}


\Abstract{
The mortgage crisis of 2007--2012 the foreclosure of some 10 million American homes.
In response, the federal government authorized three rounds of the Neighborhood Stabilization Program (NSP), a relief effort aimed at stabilizing prices and preventing further social discloation.
The program operated by purchasing, repairing, and re-selling vacant homes in neighborhoods at high risk of even more foreclosures.
Though this program resembled Democrat-supported welfare programs, there are theoretical questions about the impacts it may have had on voting.
Broadly, political economists have predicted that increased home equity leads to preferences for less social insurance, while early 21st-century American politics consistently connected less social insurance with lower taxation.
These features would suggest that the Neighborhood Stabilization Program pushed voters towards Republican candidates, especially towards self-avowed members of the Tea Party Caucus, which sharpened the taxation--social insurance connection and derided the subsidy of home mortgages.
I create a novel dataset connecting NSP target areas, 2010 midterm vote returns, home prices, and a range of demographic information in order to analyze this question.
I find some evidence that the behavior theorized did occur among races featuring a Tea Party candidate, but that the behavior stayed local to those races and was not a more broad feature of Republican support.
}

	\usepackage{booktabs}
\usepackage{longtable}
\usepackage{array}
\usepackage{multirow}
\usepackage{wrapfig}
\usepackage{float}
\usepackage{colortbl}
\usepackage{pdflscape}
\usepackage{tabu}
\usepackage{threeparttable}
\usepackage{threeparttablex}
\usepackage[normalem]{ulem}
\usepackage{makecell}
\usepackage{xcolor}
% End of CII addition
%%
%% End Preamble
%%
%
\begin{document}

% Everything below added by CII
  \maketitle

\frontmatter % this stuff will be roman-numbered
\pagenumbering{roman} % this removes page numbers from the frontmatter

  \hypersetup{linkcolor=black}
  \setcounter{tocdepth}{2}
  \tableofcontents

  \listoftables

  \listoffigures
  \begin{abstract}
  \doublespacing
    The mortgage crisis of 2007--2012 the foreclosure of some 10 million American homes.
    In response, the federal government authorized three rounds of the Neighborhood Stabilization Program (NSP), a relief effort aimed at stabilizing prices and preventing further social discloation.
    The program operated by purchasing, repairing, and re-selling vacant homes in neighborhoods at high risk of even more foreclosures.
    Though this program resembled Democrat-supported welfare programs, there are theoretical questions about the impacts it may have had on voting.
    Broadly, political economists have predicted that increased home equity leads to preferences for less social insurance, while early 21st-century American politics consistently connected less social insurance with lower taxation.
    These features would suggest that the Neighborhood Stabilization Program pushed voters towards Republican candidates, especially towards self-avowed members of the Tea Party Caucus, which sharpened the taxation--social insurance connection and derided the subsidy of home mortgages.
    I create a novel dataset connecting NSP target areas, 2010 midterm vote returns, home prices, and a range of demographic information in order to analyze this question.
    I find some evidence that the behavior theorized did occur among races featuring a Tea Party candidate, but that the behavior stayed local to those races and was not a more broad feature of Republican support.
  \end{abstract}

\mainmatter % here the regular arabic numbering starts
\pagestyle{fancyplain} % turns page numbering back on
\doublespacing
\titleformat{\chapter}[display]{\normalfont\huge\bfseries\singlespacing}{\chaptertitlename\ \thechapter}{40pt}{\huge}
\titleformat{\section}{\singlespacing\normalfont\Large\bfseries}{\thesection}{1em}{}
\titleformat{\subsection}{\singlespacing\normalfont\large\bfseries}{\thesubsection}{1em}{}
\titleformat{\subsubsection}{\singlespacing\normalfont\normalsize\bfseries}{\thesubsubsection}{1em}{}

\hypertarget{actors-motive}{%
\chapter{An Unfurling Crisis}\label{actors-motive}}

When 10 million Americans---1 of every 20 adults---lost their homes,\footnote{Martin and Niedt, \emph{Foreclosed America}.} it was clear that homeowners needed an intervention. When Countrywide, which two years earlier had serviced 20\% of the mortgages in the United States,\footnote{Seabrooke, \emph{The Politics of Housing Booms and Busts}.} collapsed in July 2008, it was clear that the real estate industry needed to change.
When cities and counties around California, Florida, Nevada, and Arizona trembled at a declining tax base, it was clear that municipalities needed a break.
Yet, none of these actors wielded much power in the response to the subprime mortgage crisis.
Unlike the Great Depression, homeowners barely organized their political power and financial institutions imposed very modest debtor-relief programs.
Additionally, the long-term trend of municipal disinvestment emphasized the need to remove neighborhood blight instead of create or subsidize housing.

In reponse, the Housing and Economic Recovery Act of 2008 (HERA) authorized the Department of Housing and Urban Development (HUD) to distribute \$3.9 billion to state and local governments for the purchase and repair of foreclosed properties.\footnote{Pelosi, ``HERA.''}
This program, later termed the Neighborhood Stabilization Program (NSP), is my focus.
In purchasing, repairing, and in most cases reselling foreclosed properties, the NSP sought to mitigate the effects of foreclosures on their surroundings.
Primarily, foreclosures tear people from their homes.
On top of this comes the stigma of being a ``deadbeat''\footnote{Dayen, \emph{Chain of Title}.} and mounds of fees---legal, cancellation, transportation-related.
But once homeowners have been forced out, the neighborhood and municipality bear the financial weight of that foreclosure.

This thesis considers two lines of analysis, the first a question of method and mechanism, the second a question of politics and historical contingency.
These two lines converge with the Neighborhood Stabilization Program amid the rise of the Tea Party.
The first attempts to add to the political science and sociology literature around the connections between housing, political preferences, and voting.
This line of analysis is motivated by the opportunity presented by the Neighborhood Stabilization Program, 2010 decennial census, and 2010 midterm elections to bridge a gap in the measurement of political behavior.
While the first two components offer unusually granular and contemporary data, the midterm elections posed a unique brand of politics to American voters.
This brand of politics was neither usual nor inevitable, an observation I take great pains to make in Chapter \ref{motive-opportunity}.
Rather, it was a product of the political and economic mechanisms investigated by the literature I engage methodologically in my first line of analysis.
The historical claims are not new; they have been fashioned directly from the work of others.
I zoom in on a particular corner of political economy and housing, however, to beef up my methodological assumptions.
The relationship between these two lines of analysis is complex but understanding it is imperative to comprehending the validity of my argument about the impact of the Neighborhood Stabilization Program.
To that end, I have provided Figure \ref{fig:analysis} to aid readers.

To nearby homeowners, foreclosures cost \$159,000 on average in decreased property values.\footnote{Immergluck, \emph{Foreclosed}, 151.}
To the city or county that envelops the property, it costs around \$50,000 for the legal and construction work to demolish and resell a vacant lot, parcels which tend to drop property values even further.\footnote{Indiviglio, ``The Housing Stabilization Program You Haven't Heard About.''}
When property values drop, a home's {[}apparent{]} equity drops too.
This figure, estimated constantly by market research firms such as Zillow and appraised periodically by taxing authorities, determines to a large extent the wealth and borrowing power of homeowners.
If the value of an already-mortgaged home rises, homeowners can refinance.
While refinancing most often takes advantage of lower interest rates or decreased principal amounts to reduce monthly payments, refinancing was used by in the lead up to the subprime mortgage crisis instead to finance renovations, service debts (eg. student, credit card, and car loans), or cash out.
Lending institutions willingly converted apparent value into real value, and as long as home values crept up, refinancing was viable.
This use of refinancing reflected a shift in how Americans approached housing, from the home as an utterly private good, to the home as an asset.
Homes stand increasingly in the position of pension plans and 401ks; a wise move can transform forty years into four-hundred thousand dollars.

These wise moves seemd to occur constantly from September 1992 until March 2006.
As Figure \ref{fig:caseshiller} shows, the intervening 14 years never once saw a drop in the industry's standard home-price metric, the Case-Shiller U.S. National Home Price Index.
As prices rose, homeowners refinanced, effectively paying back the first mortgage while taking out the second.
To investors in mortgage-backed securities (MBS), pre-payment from refinancing is usually seen as a risk, but constant and predictable refinancing added to the safety of securitization, playing into investors risk-reward calculus.
However, when prices fell---or even flatlined---this safety evaporated.
Without the increased {[}apparent{]} home equity, refinancing did nothing.
Falling prices thus meant homeowners were saddled with their current mortgage, often one whose interest rates reset after two years, climbing an interest-rate staircase for the next 28 years.
Atop the interest payments, living expenses (such as other debts) piled on to borrowers, where previously refinancing had arrived to supplement existing income with cash.
Falling prices meant homeowners chose among food, water, and shelter, if they had enough income for that to even be a decision.
\begin{figure}

{\centering \includegraphics[width=0.9\linewidth]{figure/caseshiller_1990_2018} 

}

\caption{Price (right axis) and growth (left axis) of the Case-Shiller Home Price Index, adjusted for seasonal price fluctuations.}\label{fig:caseshiller}
\end{figure}
These factors make homeowners \emph{very} sensitive to falling or stagnant home prices.
In their sensitivity, they may have voted for parties, candidates, and policies they believed would drive up home prices, such as property tax cuts, muscular code enforcement, and spacious zoning regulations.
Each of these policies reduces the resources available for non-homeowners, pitting homeowners against renters, the homeless, and anyone else without a direct financial stake in the asset.
For example, tax cuts, take they the form of directly-decreased property taxes or the homeowner tax credit, reduce the monies available to fund public goods.
In the United States, party ideology also ties higher taxes with explicitly redistributive policies, though Democrats and Republicans archetypically differ as to whether high taxes--more redistribution is good or bad.
The rigidity of these ideologies rationalizes the fear that rents extracted from higher taxation will not return to the homeowners.
Circling back to the original home-price dynamic, higher home values fund greater consumption, while lower property values and higher taxes (as they always have) limit homeowners' capacities to spend, an activity Americans very much enjoy.

When prices rose, creditor and debtor interests laid at some angle, intersecting in the particular case of safe, two-year refinancing, but divergent in the general case of refinancing that took advantage of lower interest rates.
On the other hand, falling housing prices conformed the immediate interests of creditors and debtors to one another, since delinquency and default eliminated any chance of extracting further rents.
Fears of a debt spiral triggered by falling home prices also beset cities,\footnote{Muro and Hoene, ``Fiscal Challenges Facing Cities.''} though there is some doubt that those fears were justified.\footnote{Gross et al., ``The Local Squeeze''; Lutz, Molloy, and Shan, ``The Housing Crisis and State and Local Government Tax Revenue.''}
With these interests in syzygy, why didn't homeowners, investors, or cities assert solutions, even ones that were done chiefly in their own interest?
I argue in this chapter first that homeowners remained quiet due to changes in the foreclosure process from the Great Depression, the differential character of housing versus farming, and narratives about mortgage debtors.
Second, the incongruity---spurred by the demise of savings and loans---between incentives for principals in mortgage-backed securities and agents tasked with servicing the underlying mortgages, along with fraudulent mortgage transfers, resulted in foreclosures that may have otherwise been unwanted by creditors.
Third, I point to municipal disinvestment and the trajectory of the Commerce Clause as reasons why cities were unable to handle mass foreclosures.
This critique I believe to be sufficient, though by no means necessary, to explaining why each was functionally barred from substantial action.

This argument serves dual purposes.
First, it justifes the motives of actors external to the creation of the Neighborhood Stabilization Program, to which internal actors, such as George W. Bush and members of Congress, responded.
Second, it offers a story about why external actors may not have acted decisively.
Readers should come away from this chapter understanding why federal foreclosure relief programs were necessary.

\hypertarget{homeowners}{%
\section{Why Homeowners were Unable to Organize}\label{homeowners}}

Reasons for the powerlessness of anti-foreclosure organizing can be shown through comparison with the most successful anti-foreclosure campaigns of the Great Depression.
Legislative movements require a base of public opinion and support, a mouthpiece through which opinion can be articulated, and an powerful audience to hear those articulated opinions and demands.
Foreclosed housing's base of public opinion and support was attenuated by geographic particularities of subprime mortgages which limited fora for communication, both internally and externally.
In addition, the organizers' audience was blocked partially by other demands, including the presence of mortgage servicers.
The Great Depression, however, saw these factors come together in the Midwest, where farms were physically parched and thus, financially underwater.
Farmers were positioned similar to homeowners who adopted cash-out refinancing in that they relied on their land for its income.
However, farmers in the Great Depression differed from early 2000s subprime borrowers in that farming did not \emph{supplement} wage income, it \emph{replaced} wage incomes.
In addition, access to government and stark differences in media portrayal combined with the larger impact of farm foreclosures to drive organizing that led to 27 states enacting \emph{per se} or \emph{de facto} moratoria on foreclosures.\footnote{Wheelock, ``Changing the Rules,'' 537.}
The lack of such conditions pitched the struggle for foreclosure legislation in the subprime mortgage crisis to a severe angle.

The most important differentiator between the foreclosure crisis in the Great Depression and that which preceded the Great Recession was the outsize effect of farm foreclosures.
Farms were both the workplace and home of farmers.
Unlike the foreclosure crisis in the 2000s, losing a home implied losing a job, though that job could be lost in other ways (drought, crop disease, low prices).
This fact raised the stakes for farmers to plead for relief and enlarged the macroeconomic worries with which politicians were just beginning to grapple.
Of the 100,000 farmers who lost their farms each year between 1926 and 1940,\footnote{Alston, ``Farm Foreclosures in the United States During the Interwar Period.''} the Midwest saw the highest concentrations.
Figure \ref{fig:wheelock-farms} shows the high concentration in Minnesota, Iowa, and the Dakotas, and the lesser concentration all around the Midwest.
In 1933, by far the worst year for farm mortgages, failure rates topped 3.7\%; during the rest of 1926-1940, rates were often above 1.5\%.\footnote{Wheelock, ``Changing the Rules,'' 571.}
By contrast, U.S. residential foreclosures reached 2.23\% in 2010, the worst year of the foreclosure crisis.\footnote{``U.S. Foreclosure Activity Drops to 13-Year Low in 2018.''}
In part, this is a denominator effect: the 50\% down payments and double-digit interest rates caused fewer mortgages to be demanded in the Great Depression.
In large part, however, the differences between farms and homes accounted for the scale.
\begin{figure}

{\centering \includegraphics[width=0.9\linewidth]{figure/wheelock} 

}

\caption{Figure taken from Wheelock (2008).}\label{fig:wheelock-farms}
\end{figure}
The geography of farmland created several features that made easier mass organizing.
Lower population densities meant that fewer people exercised political power over a fixed-size jurisdiction, when compared to a densely-populated district.
While this feature could not have played into U.S. Congressional politics, which are apportioned by population, it could make collusion easier in counties.
Organization at the county level was important in the Great Depression, because foreclosed properties were sold by sheriffs, elected county officials.\footnote{Fliter and Hoff, \emph{Fighting Foreclosure}.}
Compounding the numerical ease of collusion was the congruency of interests.
While farms produce a variety of crops, soil and climate particularities combine with federal agriculture policy to homogenize production locally.
In other words, Tobler's first law of geography holds: ``everything is related to everything else, but near things are more related than distant things.''\footnote{Tobler, ``A Computer Movie Simulating Urban Growth in the Detroit Region,'' 236.}
In conversation with the realities of the foreclosure crisis in the 2000s, two conclusions---one ecological, the other sociological---emerge.

On the side of ecology, crop failures occurred in conjunction with nearby farms.
The same climate or disease that killed a neighbor's crops could not be stopped at the property line.
Farmers in the Great Depression shared this feature with homeowners in the mortgage crisis: lowered property values on one side would spillover to the other side.
For both populations, spatial correlations were imperfect, as some farmers planted different crops and some foreclosures occurred among conservative borrowers or wholly-owned homes, but the spillover effect matters.\footnote{DeFusco et al., ``The Role of Price Spillovers in the American Housing Boom.''}
Falling property values decreased the value of neighboring properties, deepening the mortgage crisis for farmers in the Depression and homeowners before the Recession.
The conditions of localized, severe economic distress existed in both eras.

But this comparison did not exist between the sociological features of farming and housing.
Farming the same crops entails some degree of visiting the same market, buying the same tools, and asking the same people for advice.
Farmers met their neighbors whether they liked them or not, creating a forum to talk shop with nearby people who shared interests.
The simple fact of homeownership, however, tends to signify income, but little else, and the larger populations of suburbs facing home foreclosures meant more social and cultural institutions among which residents could choose.
The higher density and absolute size of the suburbs separated struggling homeowners from each other, while the lack of farming meant that---even if they had bumped shoulders---their mortgage finances were less likely to be topics of discussion.
While the suburban quality of foreclosures in the recent mortgage crisis could have drawn homeowners closer together through their homeowner association (HOA), HOAs were insignificant bulwarks against nearby foreclosures.\footnote{Cheung, Cunningham, and Meltzer, ``Do Homeowners Associations Mitigate or Aggravate Negative Spillovers from Neighboring Homeowner Distress?'' 87.}
And more to the point of homeowner behavior, HOA fees were some of the first payments to stop once mortgage debt piled up,\footnote{Perkins, ``Privatopia in Distress,'' 561.} suggesting that homeowners disengaged from their homeowner associations entirely.
These divergent implications for farming and suburban housing could be added to Robert Putnam's argument of a secular (in both the economic and religious senses) decline in social capital\footnote{Putnam, \emph{Bowling Alone}.} to argue that the sociological character of suburbs blocked organization around increasing mortgage delinquency and household foreclosures.

As I mentioned above, the lack of local fora was important.
In the Great Depression, not only were county sheriff offices the location of foreclosure sales, they were also the location of foreclosure sale stoppages.
Midwest farmers tried boycotting markets and sabotaging crops in transport to grab attention and force up prices, but they ``did not seriously threaten urban food supplies or raise prices or the cost of production.''\footnote{Fliter and Hoff, \emph{Fighting Foreclosure}, 4.}
Rather, farmers succeeded through intimidation tactics.
The ``ropes under {[}farmers'{]} coats {[}\ldots{]} stopped thousands more foreclosures than did'' self-organized arbitration, according to the then-president of the Farmers' Holiday Association, Milo Reno.\footnote{Fliter and Hoff, 65.}
There were more than 100 recorded instances of farmers, sometimes numbering in the thousands, packing county sheriff offices to discourage any would-be buyers, allowing the borrower to re-purchase their farm in full for sometimes as little as a penny.\footnote{Fliter and Hoff, 63.}
Direct action of this nature was nowhere in the subprime mortgage crisis; Figure \ref{fig:tradingecon} shows the spike in existing home sales after several million foreclosures had been filed.
\begin{figure}

{\centering \includegraphics[width=0.9\linewidth]{figure/united-states-existing-home-sales@2x} 

}

\caption{Existing Home Sales versus New Home Sales, 2000-2014}\label{fig:tradingecon}
\end{figure}
But organizing need not adopt the character of halting foreclosure sales; rather, the foreclosure itself could have been the point of action.
Changes to bankruptcy laws made foreclosures less defensible by making judicial hearings an opt-in rather than mandatory system.
The hearings provide both the legal forum to contest evidence, claims, and standing, as well as the social forum to support other defendants.\footnote{Dayen, \emph{Chain of Title}.}
Courts' docket sizes also elongated the period a delinquent borrower could stay in their home, during which alternative remedies may be sought.\footnote{Cheung, Cunningham, and Meltzer, ``Do Homeowners Associations Mitigate or Aggravate Negative Spillovers from Neighboring Homeowner Distress?''}
Collins, Lam, and Herbert\footnote{``State Mortgage Foreclosure Policies and Lender Interventions.''} found that even such vanilla advocacy as mailings to suggest loan modification were more effective in states with judicial foreclosures.
However, evidence regarding the change of this process over time is a mixed bag.
Cheung, Cunningham, and Meltzer\footnote{``Do Homeowners Associations Mitigate or Aggravate Negative Spillovers from Neighboring Homeowner Distress?''} argues that the rights of residents have been eroded due to the increase in states where non-judicial foreclosure is the \emph{norm}, though the actual number of states where non-judicial foreclosure is legally available has decreased.\footnote{Ghent, ``The Historical Origins of America's Mortgage Laws,'' 22--23.}

Judicial foreclosure briefly entered the national news cycle in 2010, when former homeowners alleged fraud against several large mortgage servicers, termed foreclosure mills for their prolific business.
Allied state attorneys general settled with 13 banks over their roles in using falsified titles, signatures, and documents to foreclose on 3.8 million borrowers,\footnote{Orol, ``U.S. Breaks down \$9.3 Bln Robo-Signing Settlement.''}
The fraudulent documents meant that contestation in foreclosure courts was possible, against the assumptions for foreclosures.
In fact, the New Jersey Supreme Court in 2010 ordered lower courts to stop hearing foreclosure cases due to the prevalence of fraudulent documents.\footnote{``New Jersey Courts Take Steps to Ensure Integrity of Residential Mortgage Foreclosure Process.''}
Organizers cited the political influence of foreclosure mills---particularly in hard-hit Florida, the only Sand State with mandatory judicial foreclosure---as cause for the inefficacy of activism around foreclosure fraud.
In several cases, the Florida Attorney General and elected representatives backed out of supporting investigations after meeting with employees of mortgage servicers (who were, in two cases, also employees of the Attorney General).\footnote{Dayen, \emph{Chain of Title}.}
Whatever the cause, the fraudulent mortgage documents offered a concrete opportunity for organizing that resulted in the dislocation of millions, and undermined a tradition of impeccably-kept land records that extended well before 1776.

Geographic and economic features of farming compounded its larger-scale mortgage crisis to foment conditions ripe for organizing compared to the subprime housing crisis.
The Internet forums where foreclosures were discussed and debated turned out to be silos, with arguments accumulating while few asked outside sought answers.
In contrast, farming's power to determine social interactions pushed together those in distress, joining a long line of Progressive debtor's rights movements in the Midwest.
These movements organized, articulating political demands most remarkably on March 22, 1933, when ``a caravan of two to three thousand farmers descended upon St.~Paul from southern Minnesota, in an astonishing array of antediluvian automobiles, and swarmed over the capitol.''\footnote{Fliter and Hoff, \emph{Fighting Foreclosure}.}
This swarm presaged the unanimous passage of the Minnesota Moratorium Act, a key piece of state mortgage legislation whose constitutionality would be upheld in \emph{Home Building \& Loan Association v. Blaisdell},\footnote{``Blaisdell.''} paving the way for further states to enact statutory protections for mortgagors during the Great Depression.
In contrast, each call for mortgage moratoria in the subprime mortgage crisis was met with consternation, as scholarly opinion soured on \emph{Blaisdell}.
To see why state resources went unspent, I look again to the legal history of economic regulation in the United States, and tour briefly the trend of municipal disinvestment.

\hypertarget{cities-states}{%
\section{Why City \& State Administrations Could Not Handle Foreclosures}\label{cities-states}}

The Commerce Clause was written, and did develop, with the express purpose of pre-empting states' right to economic legislation.
Indeed, it's development in legal precedent followed such a pattern in the 1800s and 1900s, upholding regulation based on the Commerce Clause as constitutional essentially whenever business touched multiple states.
This criterion is important to foreclosure policy because I analyze in later chapters the spillover effect of foreclosures.
Spillover effects respect no legal boundary, and, as such, found themselves under federal jurisdiction by way of the Commerce Clause.
Later in the twentieth century, taxpayer reform organizations yanked the reins of state and municipal budgets.
This influence combined with the altered expectation of federal policy regulating economics, leaving only growth-oriented tax incentives and zoning regulations in the regulatory toolbox of cities and states.
This turn away from local economic regulation starved states and cities of the resources needed to handle foreclosures.

Features of the U.S. Constitution were interpreted by the Supreme Court to justify federal action, which is partially responsible for the shift in spending from states and municipalities to the federal government as seen in Figure \ref{fig:spending-shift}.
While scholarhsip has elaborated (and debated) Charles Beard's story of private interests in the Constitutional Convention, Beard\footnote{\emph{An Economic Interpretation of the Constitution of the United States}.} retains its value by its analysis of primary sources.
In it, Beard argues that the United States has a long history of enforcing the rights of creditors over the sovereignty of states.
Linking uprisings such as Shays Rebellion to foreign credit demands and the interests of individual urban bondholders, he suggests that the Constiution can be seen broadly as a struggle between farmers and bondholders, not unlike Depression-era rhetoric between Midwest farmers leaden with debts and their Eastern creditors.

The primary result of this struggle was the Contract Clause:
\begin{quote}
No State shall {[}\ldots{]} coin Money; emit Bills of Credit; make any Thing but gold and silver Coin a Tender in Payment of Debts; pass any Bill of Attainder, ex post facto Law, or Law impairing the Obligation of Contracts\footnote{Madison, ``The Constitution of the United States.''}
\end{quote}
The Contract Clause limited state powers for economic regulation in three main ways.
First, it removed from states the ability to print money, and, thus, to overprint money.
While a central bank had not been created yet, the Contract Clause ensured that no state could chip away at debt by deflating their currency.
Two, the Clause made standard payment in metal, limiting their purchasing power.
Third, this bit of the Constitution, in its statements regarding ``ex post facto Law'' and ``impairing the Obligation of Contracts'', removed from states the ability to cancel debts by removing the ability to cancel any contracts at all.
Debt cancellation or reduction was a primary aim of uprisings like Shays Rebellion and, more softly, wealthy planter political connections.\footnote{Beard, \emph{An Economic Interpretation of the Constitution of the United States}.}

In the Great Depression, the meaning of the Contract Clause was challenged, as mentioned, in \emph{Blaisdell}.
But scholarly opinion had soured on Chief Justice Charles Hughes' words that, ``While emergency does not create power, emergency may furnish the occasion for the exercise of power.''
Richard Epstein, one of the Chicago School's leading legal scholars, writes that, ``the police power exception has come to eviscerate the contracts clause,''\footnote{Epstein, ``Toward a Revitalization of the Contract Clause,'' 738.} and he is not alone.
Indeed, Tim Geithner's approach to the foreclosure policies emphasized legality as one of its tenets, and pointed to state usage of the Contract Clause as a grey area.\footnote{McNamara, ``Yale Program on Financial Stability Interview.''}
Constitutional limits on state (and thereby local) economic regulation were bolstered by scholarly backlash and the significant influence of originalism and textualism on the Court.
If Beard's interpretation of the Contract Clause is correct, then conservative justices would likely have blocked any such attempts to establish the foreclosure moratoria imposed during the Depression.

The Contract Clause's intent, and the recent turn back towards honoring such intents, has limited states from pursuing such monumental measures as moratoria.
But within the scheme of regulating business dealings, the Clause left states with great room to move.
This range of movement was restricted further by the twentieth century interpretation of the Commerce Clause.
While both the Contract and Commerce Clauses reflected Hamiltonian designs on state sovereignty, they have been invoked differentially by Republicans and Democrats, situating themselves on an axis whose ends represent pro- and anti-business interests.
Defenders of the Contract Clause's intended usage prioritize the right to contract as a pre-political right over the right of local government.
Defenders of the Commerce Clause prioritize the federal government's capacity and judgment to regulate business that stretches across state lines over the right of local government.
Unlike the Contract Clause, whose recent judgments and scholarship point to unconstitutionality of state-passed foreclosure moratoria,\footnote{Note that the New Jersey ``moratorium'' was neither a legislative act nor a blanket moratorium. It affected the litigation of foreclosures and referenced the validity of evidence itself, which would hypothetically be inadmissable regardless of the order.} the Commerce Clause has been granted broad powers, with conservatives on the Court only tinkering at the edges of its range of freedom.

For instance, where \emph{Blaisdell} squeaked by with a 5-4 decision, \emph{Wickard v. Filburn}\footnote{``Wickard V. Filburn.''} unanimously upheld the authority of the Agricultural Adjustment Act to regulate private acts---ones that had never seen a marketplace---which substantially effected the business of another state.\footnote{``Wickard V. Filburn.''} capped five to seven years of the Commerce Clause's usage to authorize New Deal policies.
If the powers extended down to peoples' private properties, what jurisdiction did the federal government \emph{not} possess?
For the subprime mortgage crisis, this history of Commerce Clause--facilitated pre-emption meant that the federal government was expected to the intervene during economic crises.
All that was needed to authorize pre-emption was the substantial effects test employed in \emph{Wickard}.
Let me be clear on this point: pre-emption via the Commerce Clause did not crowd out state investment; in fact, there is ongoing debate as to whether federal dollars \emph{increase} state investment.\footnote{Cf. ``flypaper effect'' in public finance scholarship.}
Rather, the New Deal wielded the Commerce Clause to achieve its own ends.
Over time, the the substantial effects test came to mean that even supposedly-private activities could be federally regulated.
Home construction, and possibly foreclosure, given their spillover effects, certainly hit this threshold.
Effectively, I argue that a coincidence of responsibility, and a history of the federal government taking on responsibility in times of crisis, meant that states presumed the ball was in Washington's court.
This stands in contrast to the Contract Clause, the interpretation of which has limited state options.
\begin{figure}

{\centering \includegraphics[width=0.9\linewidth]{figure/spending-shift} 

}

\caption{While secular growth is present at all levels of government, note the spikes during the Great Depression and Recession, and of course in wars.}\label{fig:spending-shift}
\end{figure}
Running alongside these legal developments and historical acts by the federal government, taxpayers had been throttling local revenues.
The tax revolts, beginning with California in 1978, play a crucial part in the ideological formatting of reactions to the subprime mortgage crisis, but for now I will focus on their effects on revenues.
Tax revolts, and tax reform movements more generally, have been significant political forces at all levels of American politics.
Whether in regards to a specific tax or taxation generally, taxpayer movements focus on limiting or rolling back tax rates and taxable activity.
Their efficacy, however, has been limited; in many cases, revenue and spending provisions can be circumvented to meet legislative and administrative desires.\footnote{Gross et al., ``The Local Squeeze.''}
So, while specific taxes have been reined in by voters, the level of taxation is difficult to wrangle.

The one area secured by taxpayer reform efforts have been measures that require legislative supermajorities or popular referenda to raise rates.
For instance, the original California movement succeeded in restraining property tax rates from climbing above 1\% and required a legislative supermajority equal to that needed to amend the state constitution in order to raise special taxes.
Provisions such as this have been successful at limiting property tax receipts,\footnote{Kioko and Martell, ``State-Level Tax and Expenditure Limits.''} effects which bear directly on state and local abilities to raise revenue from housing.
In housing-rich states---such as California, Florida, Nevada, and Arizona---that saw so much of their housing stock go vacant, this feature puts a double-bind on states and localities.
It implements a strongly pro-cyclic revenue structure that conflicts with the anti-cyclic need for government investment,\footnote{Keynes and Krugman, \emph{The General Theory of Employment, Interest, and Money}.} though such a gap would not grow until assessments had revalued property in the face of the housing bust.
In 2007, 2008, and 2009, this feature likely acted as the lower arm of a pair of price scissors to homeowners---though they were \emph{very} rusty, unable to close completely since property taxes rarely top 2\%---holding stable as housing prices plummeted.
There is little research of this effect on municipal finances, but more recent scholarship points to large decreases beginning in 2009 or 2010 and extending as late as 2013.\footnote{Chernick, Reschovsky, and Newman, ``The Effect of the Housing Crisis on the Finances of Central Cities''; Gross et al., ``The Local Squeeze.''}
The size of this lag may account for the lack of consensus on the topic, with 2011 research by the Federal Reseve Board of Governors denying significant fiscal effects of foreclosures.\footnote{Lutz, Molloy, and Shan, ``The Housing Crisis and State and Local Government Tax Revenue.''}
At any rate, while property taxes did strain municipal budgets, and while cities did not feel the effects until after the foreclosure crisis elicited policy responses, anxiety over the coming problems mounted.\footnote{Dennis, ``Falling Home Values Mean Budget Crunches for Cities''; Saulny, ``Financial Crisis Takes a Toll on Already-Squeezed Cities.''}

I have outlined structural forces that limited the legal and fiscal capacities of states to intervene in private activities, even when such activites have heavily public effects.
While the structural forces clearly affect much more than housing, the funding of American states and towns is heavily indebted towards property taxes, bringing one of every three municipal dollars.\footnote{Gross et al., ``The Local Squeeze,'' 1.}
While property taxes had not yet declined when the foreclosure crisis hit its lows, anxieties were rising, and taxpayer reform movements had already bit large chunks out of the ability to raise money.
These forces starved states and cities of the resources necessary to handle foreclosure.
Endowed only with powers to issue zoning regulations, tax \emph{incentives}, and other growth-oriented policies,\footnote{Stein, \emph{Capital City}.} cities and states were less able to handle the wave of foreclosures that pushed one of every twenty American adults out of their home.\footnote{Martin and Niedt, \emph{Foreclosed America}, 5.}

\hypertarget{banks}{%
\section{Why Investors were Unable, and Banks Unwilling, to Limit Foreclosures}\label{banks}}

The separation of investors from mortgage servicers constituted a separation of ownership from control.
Following the neoclassical literature,(Berle \& Means) this separation led to perverse incentives: while investors lost money from foreclosures, the trustees of mortgage-backed securities gained money from foreclosures.
Securitization severed the communicative link between investing and mortgage servicing.
When their interests were pitted against one another, the legal priority fell to the trustee, in whose interest it was to foreclose.
Before the separation of ownership from control in mortgages, a bank could decide---as many did in the Great Depression---not to foreclose in spite of its legal right.

Securitization is a complicated process, pictured in Figure \ref{fig:immergluck} legally incorporating the collection of mortgages that back a residential mortgage-backed securities (MBS).
First, a mortgage broker secures the original agreement between mortgagor and mortgagee.
For the homeowner, this is most often all they ever see, and for the broker, this had been true until the 1970s.
At that time, the Federal National Mortgage Association (Fannie Mae) and Federal Home Loan Mortgage Corporation (Freddie Mac) were spun off by the federal government into the government-sponsored entities that existed until the mortgage crisis.
Fannie and Freddie brokered mortgages to prime borrowers, those considered least likely to default.
Yet the low risk could only the price of credit (aka interest rates) by so much.
Securitization offered a second layer of insulation from the unpredictability of individuals, leading to lower risk, and lower interest rates, making mortgages attractive to propsective homeowners.
In this case it was the knowledge that investors would accept low interest rates---made acceptable by global disinflation and a massive supply of savings\footnote{Schwartz, \emph{Subprime Nation}.}---which facilitated such attractive interest rates.
After brokering a mortgage, Fannie Mae or Freddie Mac would then take a few thousand other mortgages and combine them into a mortgage pool.
Later these pools would contain mortgages brokered by dozens, perhaps hundreds of firms, who would immediately sell them to originators.
These two roles, broker and originator, seprated over time, and created the initial problems whereby bad loans were good business.\footnote{Immergluck, \emph{Foreclosed}, 103.}

After pooling, the securitizer would then establish a special purpose vehicle (SPV) with the sole purpose of holding mortgages and issuing financial securities.
Instead of issuing pieces of individual mortgages, the vehicle issued pieces of itself, a self that was fully composed of the cashflow from mortgages, in the form of claims on the pool's profits.
These profits would then be divided into tranches corresponding to different levels of risk, and thus, reward for the investor.
Legally, the tranches specified who would be paid first and who would suffer losses first, with low-risk investors insulated from both heavy gains and heavy losses.
I use the word profits deliberately: while the claims were considered ``pass-through'', meaning that revenue from mortgages was attached legally to payments to investors, there were costs to securitization.
The special purpose vehicle took on the mortgage pool itself, thus obligating it to service the underlying loans.
SPVs, with no legal employees, designated companies in their founding documents to act as servicers, scheduling fees for particular services.
While investors could open informal lines of communication with those tasked by securitizers to administer payments, there was no formal mechanism by which investors could exercise the kind of shareholder democracy that had become so near and dear to their hearts.
\begin{figure}

{\centering \includegraphics[width=0.9\linewidth]{figure/immergluck} 

}

\caption{Taken from Immergluck (2011).}\label{fig:immergluck}
\end{figure}
The practical particularities of SPVs thus severed any remaining lines of communication by which investors could express their interests.
While interests would not have been \emph{congruent} with homeowners, they may have been coincident.
During the Great Depression, Prudential, then the largest holder of farm debt in the country, ceased collection on farm mortgages.\footnote{Fliter and Hoff, \emph{Fighting Foreclosure}, 65.}
Foreclosure ensures that a mortgage debts cannot be collected on, as it cancels the contract linking debtor with creditor.
Usually, foreclosure may be a good way to hedge against nonpayment, and it was in this framework that servicers were promised payments for enforcing property rights by foreclosure actions.
But in a housing crisis, when flattening home values give way to falling, and then plunging home values, adding more supply to a housing market only serves to decrease prices more.
Securitization provided agents, the mortgage servicers, to escape the interests of the principals, investors.
By removing the principal from the process, MBS securitization granted the agent free{[}r{]} reign.

While foreclosures were in the interests of the mortgage-servicing agents,\footnote{Goodman, ``For Mortgage Servicers, an Incentive Not to Help Homeowners.''} they were less likely to be in the interests of investors.
Depending on their tranche in an MBS, some investors would have actually \emph{preferred} to keep loan terms stringent in order to increase risk, and thus reward.
These inter-investor squabbles meant that servicers that acted along informal lines of request ``may be seen as instigating interparty litigation.''\footnote{Immergluck, \emph{Foreclosed}, 103.}
I do not argue that investors necessarily would have ceased foreclosure in times of crisis, but that, like banks in the Depression that were long in their own investments, they may have added extra time for payment had they the organizational mechanisms to do so.
It is likely that the historical reality of American residential mortgage-backed security would have prevented this anyway: foreign capital held 20\% of residential MBS in the United States.
Where thrifts and even large retail banks in the Great Depression had a choice, the clauses in Pooling \& Servicing Agreements ensured that the information flow only went in one direction, from mortgage to investor.
Securitization, in separating ownership from control of the underlying mortgages, made bad loans good business, as detailed by Michael Lewis' bestseller \emph{The Big Short}, but an overlooked aspect of securitization was its alignment of interests \emph{after} loans had gone bad.

I have identified structural factors that explain why three of the four direct interests in housing---homeowners, cities and states, and investors in residential mortgage-backed securities---were unable or unwilling to organize large foreclosure relief efforts.
In the first and third cases, actors were unable to organize due to differences in the political economy (and often, geography) of the subprime mortgage crisis when compared to the Great Depression.
In the case of cities and states, federal pre-emption and strained finances limited their legal and fiscal capabilities.
None of these three arguments deliver the kind of logical suplex blow that they deserve, but they serve to quickly explain why federal intervention in foreclosure relief efforts was necessary.
My next chapter will describe how this motivation crystallized into the raft of relief policies that the Bush administration delivered, including the Neighborhood Stabilization Program.
It will then detail how the program was administered.
I will bookend these discussions with theoretical and historical understandings of American party politics, for the crucial link between taxes, housing, and party politics aids explanations of why the Bush administration and Congress did what they did.

\hypertarget{motive-opportunity}{%
\chapter{Connecting Home Prices to Voting}\label{motive-opportunity}}

\epigraph{We're creating...an ownership society in this country, where more Americans than ever will be able to open up their door where they live and say, welcome to my house, welcome to my piece of property.[@klein2008disowned]}

The Housing and Economic Recovery Act (HERA) countered the unfurling crisis with a ream of policies aimed, variously, at mitigating spillover effects of the subprime mortgage crisis.
This aim ran parallel with the Bush administration's ``ownership society'', which added moral heft to the legal responsibilities associated with private property.
By mitigating spillover effects, the Neighborhood Stabilization Program would work in service of this doctrine.
The NSP would attempt to isolate homes from the ``animal spirits''\footnote{Keynes and Krugman, \emph{The General Theory of Employment, Interest, and Money}.} of tumultuous housing markets and ensure that the success or failure of a person's finances depended on their own actions.
However, despite the ideological congruence between the NSP and Bush's ownership society, the Neighborhood Stabilization Program departed from the Republican Party by privatizing government money for less (or even negatively) wealthy people.
Recently in the United States, both Republican and Democratic parties have connected this behavior---support for programs that transfer taxpayer money to less affluent people---with higher taxation, and with the post--New Deal Democratic Party.
So, while the policies fit with the Bush administration, their ideological dimension tended more towards the Democratic Party.

But, as mentioned early in Chapter \ref{actors-motive}, opposing the leftward pull of ideology was the rightward push of rising wealth.
While higher levels of wealth and income are colloquially associated with conservatism, but there is little research on how asset ownership pushes people towards their political decisions.(BRING IN EMBOURGEOISIEMENT HERE?)
And, historically, the political effects of taxation were not a given: after World War II, taxation was not a weighty political issue,\footnote{Campbell, ``How Americans Think About Taxes,'' 160.} let alone a wedge issue.
Rather, responses to California's Proposition 13 and reverberations of Reaganite rhetoric led Republicans ``to define themselves and their party in opposition to taxes.''\footnote{Martin, ``Welcome to the Tax Cutting Party,'' 128.}
Politicians of all stripes followed, with Third Way Democrats, led by President Bill Clinton, accepted tax cuts in their successful political bids; it was only in the 2000s that Democrats united (more or less) in warning against lower taxation.

The mortgage crisis hit after this crystallization of interests: a moment in American politics when the two parties agreed on the content of their debates.
Specifically, they agreed that---along with wars in the Middle East---taxation and visible, means-tested welfare were the political issues of the day.
The only difference were the moral valences each party attached to the issues.

This connection between taxation, welfare, and party platforms came to a head in the late 2000s.
After the mortgage crisis ushered in a financial crisis and the Great Recession, economic policy adopted a militant vocabulary and assumed powers fit for a war.
In response, grassroots anti-tax movements (thoroughly astroturfed by conservative donors and non-profits) took up equally rhetorical arms, with CNBC reporter Rick Santelli sounding the battle cry of the Tea Party Movement.

The Tea Party Movement sharpened the debate over taxes.
What had been salient in the decades before the trauma of the financial and mortgage crises re-emerged as the guiding issue of the Republican Party.
The salience of taxation, and the formalization of candidates as members of the Tea Party Caucus, made the theoretical underpinnings legibile to voting analysis.
Without the self-professed and much-discussed focus on taxation during the 2010 midterm elections, analyzing voters' awareness of candidates' tax positions would be required in order to effectively argue that voting behavior ran with tax policies.
The focus on taxation, and the clear connection between expenditures on visible welfare programs and higher taxation, make possible the aggregation of individual tax preferences into party voting.

In this chapter, I dive into how the Neighborhood Stabilization Program fit ideologically with the Bush administration and Republican Party between 2008 and 2010.
I argue that the NSP, by virtue of its targeted recipients, hewed more closely to the ideology of the Democratic Party than that of the Republican Party between 2008 and 2010.
Further, the snugness of this fit was not a given, but was contingent on a tax protest movement like that of the Tea Party.
American party politics were crucial to corralling the effects of foreclosure relief programs into distinctions legible to vote analysts.

\hypertarget{the-neighborhood-stabilization-program-ideologically}{%
\section{The Neighborhood Stabilization Program, Ideologically}\label{the-neighborhood-stabilization-program-ideologically}}

Section 2301(a) of the Housing and Economic Recovery Act of 2008 appropriates \$3,920,000,000
\begin{quote}
for assistance to States and units of general local government (as such terms are defined in section 102 of the Housing and Community Development Act of 1974 (42 U.S.C. 5302)) for the redevelopment of abandoned and foreclosed upon homes and residential properties.
\end{quote}
Due to the size of the bill passed and the extent of modifications by a range of members, it is unclear who inserted the provision.
However, a select few news sources picked up on the insertion, including a Reuters piece which claimed ``The White House had originally opposed a provision that offers \$4 billion in grants to states to buy and repair foreclosed homes.''\footnote{Pelofsky, ``Bush Signs Housing Bill as Fannie Mae Grows.''}
The opposition by the Bush administration of the \$4 billion outlay, a mere 1.3\% of the \$300 billion ensured to Fannie Mae, indicates that the amender likely belonged to the Democratic Party.
Albeit brief, this opposition---the only opposition mentioned in an article awash in hundred-billion-dollar outlays---is surprising given the NSP's tiny size and the measures taken to distance the federal government from political (and legal) liability.

The Department of Housing and Urban Development (HUD) administered the funds through its Community Development Block Grant program, a core piece of HUD's community investment apparatus.
By routing the funding through a pre-existing program, the bill's writers sought to leverage established lines of communication between HUD and the several states.
This provision also avoided long-term commitments by HUD to far-flung localities, pushing the purchase and selling of properties to an arms-length.
But, while the bill collected information and decision-making at the local level, its compliance regulations were still onerous.
Dan Immergluck (2015) argues that ensuing language in HERA, which limited localities to purchase properties at a discount from market price, lessened the program's efficacy.
Much like with the market for used cars, houses at a discount are always already more likely to be lemons, since the seller wouldn't take less than market price for an above-average quality home.
Beyond that, federal oversight of localities was intense, as the administration took an extremely risk-averse approach to mismanagement of funds, fitting the image pushed by conservatives in Bush's party of housing programs as loci of graft and inefficiency.\footnote{Immergluck, \emph{Preventing the Next Mortgage Crisis}.}

Funds were originally calculated by a need-based formula factoring in rates of foreclosure, subprime borrowing, and payment delinquency in order to address areas ``identified by the State or unit of general local government as likely to face a significant rise in the rate of home foreclosures.''\footnote{Pelosi, ``HERA.''}
In doing so, the legislation aimed to use the buying power of the federal government to bid up prices in neighborhoods by taking housing stock off the market (reducing supply), recording purchase prices close to market levels (benchmarking nearby appraisals and sales), and reselling said housing stock after maintenance and repairs (increasing quality).

But the political import of this plan is difficult to understand in the vacuum of its administration and guiding legislation.
At the time, dispossessed homeowners and their coalitions called for principal reductions, a common decision in courts of equity and bankruptcy cases, or immediate relief to homeowners currently in default.
Rather than direct cash transfers, which would have enhanced individual freedom by loosening household budget constraints, or pass-through payments to debt holders, which would have promoted rent-seeking by those debt holders, the Neighborhood Stabilization Program advocated for indirect bumps on home equity.

The NSP attempted to cosset homeowners from endogenous shocks, recognizing the social dislocation caused by such a stupid destruction of private wealth.
While some homeowners could be painted as the kind of irresponsible, purely-speculating ``deadbeats'' characterized in Dayen (2016) and Michael Lewis' \emph{The Big Short}, it is important to remember that the majority of those foreclosed upon were prime borrowers.\footnote{Ferreira and Gyourko, ``A New Look at the U.S. Foreclosure Crisis.''}
Further, the distress of prime borrowers came somewhat after that of subprime borrowers, as seen in Figure \ref{fig:distress}.\footnote{Though that lag was partially because there was more ``fuel left to burn'' in the sense that the number of subprime borrowers, already only 20\% of the market, had already been slashed.}
These facts suggest that neighborhoods with prime borrowers contracted the foreclosure contagion from subprime borrowers, which seems, 12 years later, to be proximately true.(TOOZE? MIAN \& SUFI?)
The political stigmata came when XXX fingered corner-cutting subprime homeowners for forcing their responsible, well-to-do homeowners into foreclsoure, rather than their collective panic, lax credit markets, or negligent mortgage brokering.
\begin{figure}

{\centering \subfloat[Total foreclosures and short sales by owner type\label{fig:distress1}]{\includegraphics[width=0.45\linewidth]{figure/ferreira_absolute} }\subfloat[Probability of home loss by owner type.\label{fig:distress2}]{\includegraphics[width=0.45\linewidth]{figure/ferreira_distress} }

}

\caption{Measures of homeowner distress timing.}\label{fig:distress}
\end{figure}
Protected from the shocks to home prices (and thus home equity), the success or failure of borrowers' real estate investment would then depend purely on their payments.
To maintain the idea of a meritocratic path to the American Dream, from time to time the route must be cleared of such hazards.
Note that the Neighborhood Stabilization Program did not aim, like its administrative superstructure does, to develop communities in any meaningful way.
Rescuing communities from visual blight were incidental effects of the policy and tied only to blight's impact on home prices.
Rather, the NSP aimed at the more foundational goal of saving the plunging home equities of several million individuals.

This individuation, this buffer from the market, is a privilege long constructed for the American home by law and policy.
Specifically, the private-public distinction is made much sharper for homeowners than for renters.
Consider for instance 4th Amendment cases, where {[}English{]} property law still dominates legal interpretation (despite the impact of \emph{Katz v. United States}, which expanded what the 4th Amendment shielded to include more person-centric concepts).\footnote{Stern, ``The Inviolate Home.''}
Or consider the mortgage interest deduction, a large subsidy applied to owner-occupied properties.
The history of American housing policy---both as centralized legislation and decentralized legal interpretation---has conferred on homeowners (who, in the case of single-family housing, are also land owners) additional rights when compared to renters, expanding the legal extent of their individual selves.

These privileges made up much of the popular appeal of George W. Bush's ``ownership society''.
While it lived a larger life in the media than in Bush's own remarks, the philosophy can explain much of Bush's domestic policy, and should therefore be taken seriously.
The definition is almost self-evident: an America where as much is owned and controlled by individuals as is feasible.
Bush explained the logic guiding this idea by claiming that ``The more ownership there is in America, the more vitality there is in America, and the more people have a vital stake in the future of this country.''\footnote{``America's Ownership Society.''}
With this claim came a raft of policies demonstrating the promise of an ownership society.

Under this capacious roof, the administration housed health savings accounts, more individuated retirement options, and policies aimed at boosting homeownership.
These policies included the American Dream Downpayment Initiative (fairly self-explanatory) and the Single-Family Affordable Housing Tax Credit.\footnote{``Homeownership Policy Book.''}
By encouraging single-family homes in the ``Nation's inner cities'', thinning the density of apartment complexes and tenements, this last provision hearkened back to New Deal marketing, which idealized the home as a Romantic ``refuge from urban corruption''\footnote{Stern, ``The Inviolate Home,'' 910.} associated with a high-density cityscape and the web of problems spun around public housing complexes.
Bush's ownership society provisions also included a ``tripling of funding''\footnote{``Homeownership Policy Book.''} for a program that swapped personal or volunteer labor for housing assistance, known in low-income housing development as ``sweat equity''.
These programs in aggregate, pushed a narrative of meritocratic access to consumer credit spurring individual ownership.
The NSP, in turn, slotted into this story by insulating responsible (i.e.~those who have not yet defaulted) debtors from wily housing markets.

Such a conformity with Bush's animating philosophy, despite the program's brief opposition by the president, places the Neighborhood Stabilization Program in murky territory with regard to party affiliation.
If voters were unable to locate programs like the NSP within Democratic or Republican platforms, then there would be no clear reason why pro--home price policies translate into Republican votes.
But while the values cherished by Bush's ownership society aligned ideologically with American conservatism, its distinctive policies were afield of the Republican platform.
Part of this separation was because Bush's ownership society was not too different than the messages had pushed by presidents on both sides of the aisle since at least the Great Depression.\footnote{Stoller, ``The Housing Crash and the End of American Citizenship.''}
Rather, the policies looked like a {[}more{]} moralistic version of Democratic plans for affordable housing and access to credit.

\hypertarget{shifting-priorities-in-the-grand-old-party}{%
\section{Shifting Priorities in the Grand Old Party}\label{shifting-priorities-in-the-grand-old-party}}

Republican antipathy to taxation and opposition to visible, continuing welfare assistance crowded out other domestic policy issues, like the politics of debt that had been so visible to Midwest farmers in the 1930s, Massachusetts farmers in the 1780s, and opponents of the Cross of Gold in the late 19th century.
Instead, what talk there was regarding debt resulted from a more general concern over asset prices.
This talk proceeded from the expansion of finance's place in the economy and culture of the United States.
While bankers had always been handsomely paid, it was not until the liberalizations of the Reagan era that its ranks swelled in quantity and individual wealth.
These included passage of the Garn--St.~Germain Depository Institutions Act of 1982---which Paul Krugman held responsible for the savings and loan crisis by granting permission to private lenders for the sale of adjustable-rate mortgages\footnote{Krugman, ``Reagan Did It.''}---but I pay particular attention to the exaltation of multinational creditors in mortgage markets.
A year before the Garn--St.~Germain Act, Reagan enacted his signature reform, the Economic Recovery Tax Act of 1981, which allowed troubled savings and loan institutions to sell home mortgages and consequently write down their tax burdens.\footnote{Lewis, ``The Fat Men and Their Marvelous Money Machine.''}
This provision opened the floodgates for private-label mortgage securitization; where savings and loans had dominated the private market, investment banks could now cash in.
By definition, it was these creditors that enabled the expansion of mortgage-backed securities to subprime lenders, and with them, significant economic growth.

Between 1981 and 2007, outstanding residential mortgage debt---shown in Figure \ref{fig:hhdebt}---swelled from 37\% of US gross domestic product to 82\%, more than doubling in relative size.\footnote{Board of Governors of the Federal Reserve System (US), ``MDOTPFP''; Board of Governors of the Federal Reserve System (US), ``MDOTPNNRP''; Board of Governors of the Federal Reserve System (US), ``MDOAH''; U.S. Bureau of Economic Analysis, ``GDPA.''}
On the other side of the largest pool of consumer debt in the world stood an even larger pool of capital.
While the rampant foreign direct investment (FDI) of Japan and Germany are often noted in economic histories of the 1980s, potentially more significant were the destination of such capital flows.
Herman Schwartz disaggregates the flows of FDI entering and exiting the United States to explain why, despite the large net debt position of America, the country was not hamstrung by austerity provisions but actually received net positive income.
His answer is that the consumption-driven American economy leveraged the credit made cheap by global disinflation to extend its own resources and drive aggregate demand.\footnote{Schwartz, \emph{Subprime Nation}.}
Global disinflation resulted partially from the sheer supply of credit, in turn making the interest rates offered by US mortgage-backed securities a viable alternative to dampening rates of return on traditional investments.
This process fueled itself for a while: lower interest rates meant that buyers could afford more expensive homes on the same monthly payments, bidding up the prices of homes; when nearby home prices appreciated, homeowners could refinance, paying off the first mortgage, marginally lowering the risk and thus interest rate for the next mortgagee in the market.
This process played out in every country able to profitably securitize mortgages, but the Reagan administration's reconfiguration of credit enabled the United States to reap the world's largest benefits---Schwartz credits this system with driving the United States' differential growth above the rich Organization for Economic Cooperation and Development (OECD) countries between 1991 and 2005.\footnote{Schwartz, 4.}
\begin{figure}

{\centering \includegraphics[width=0.9\linewidth]{figure/residential_debt_gdp} 

}

\caption{Outstanding US residential mortgage debt as a percentage of US gross domestic product.}\label{fig:hhdebt}
\end{figure}
More than good macroeconomic policy, the growth buffered homeowners from wage stagnation or job loss altogether.
In conjunction with liberalized finance, norms and regulations were repealed that had protected the Fordist labor compact linking secure employment with secure mortgages.
While these repeals have wrought huge blows to bargaining power, their perceived impact was aided---and perhaps eclipsed---by the spatial and international movements of manufacturing jobs away from the organized mid-Atlantic and Midwest, to the unorganized American South, and ultimately to the Global South.
In any case, labor markets became increasingly volatile in the United States, necessitating some form of insurance.
Matt Stoller argues that the exaltation of finance, subordination of labor, and mitigation of social insurance policies can be viewed macrosocially as a renegotiation of the American social contract inaugurated by the New Deal.\footnote{Stoller, ``The Housing Crash and the End of American Citizenship.''}
Where businesses had profited on the aggregate demand generated by collectively-bargained salaries and the commensurate long-term commitments by employer and employee, the new asset-backed political economy\footnote{Note too the growth in equity holdings by Americans. In 1962, 1 in 7 households held more than \$2,000 in stocks; by 1983, with the growth of IRAs and defined-contribution plans, nearly 1 in 4 households held more than \$2,000 in stock (1992 dollars), Poterba et al., ``Stock Ownership Patterns, Stock Market Fluctuations, and Consumption,'' 321} discarded with the need for stable, steadily increasing salaries.(CHECK WITH SCHWARTZ ON THIS. HIS COMMENT SEEMS TO BE IN TENSION WITH WHAT HE TAUGHT IN PLCP 4315 ABOUT THE NORMALCY OF SLACK WAGES)
These jobs, salaries, and benefits then acted as dragnets on profits, ultimately to be jettisoned at the behest of shareholders.
In return, the asset-backed social contract offered cheap credit and the Minskyan promise of speculative returns to supplement, or supplant, income.

By locating this process in the Reagan administration, I have obviously extended beyond my self-imposed 1991 start date for continually-increasing home prices.
The point I make here is that the infrastructure for that incredibly-long price surge was built a decade before.
Then, in the 1990s, the endogenous ability of speculative home buying ramped up.
I point to somwhere in the later part of that decade as an inflection point in the Minsky cycle (pictured in Figure \ref{fig:minsky}) between speculative and Ponzi financing, a point before which home building and buying was foundational to this new ``social contract'' but still directly dependent on the income swings and employment patterns of millions (what consultants call ``market fundamentals'').
Despite 1991's brief negative sojourn, this social contract was contingent on rising home prices,\footnote{Stoller, ``The Housing Crash and the End of American Citizenship.''} and leveraged by Bush the elder as well as Bill Clinton in their presidencies.
Housing, often couched in the language of the American Dream, garnered vaguely nonpartisan support at the national level: while there were differences between Republican and Democratic housing policies, their approaches were united by a vocabulary of taxation and credit, not debt.
Despite the rich history of debt politics among American farmers, parties focused on expanding access to credit and limiting the toll to government expenditures, rather than limiting the burden of debt and expanding government revenues.
\begin{figure}

{\centering \includegraphics[width=0.9\linewidth]{figure/minsky} 

}

\caption{Minsky cycle, theorized by Hyman Minsky.}\label{fig:minsky}
\end{figure}
To be sure, this vocabulary was a rational response to the rising home prices---so long as markets showed signs of growth, debt relief would be a minor issue, while tax reductions and access to credit extended individual budgets.
But it was not uniquely rational: dependable debt relief could, if annualized, be indistinguishable from a similarly-sized mortgage tax credit; after all, with all the easy credit, there was certainly enough debt to go around---in addition to mortgage debt equaling 82\% of gross domestic product, mortgage debt service payments in 2007 increased 75\% over 1981 levels.\footnote{Board of Governors of the Federal Reserve System (US), ``MDSP.''}

To summarize, the Neighborhood Stabilization Program sought to insulate homeowners from the market by using the buying power of the federal government to fix local home prices.
This program, despite the initial opposition of the Bush administration, fit with George W. Bush's ``ownership society'', which emphasized private, family-owned property as both a social and economic doctrine.
But Bush's ownership society, while conservative, helped established and would-be homeowners manage and mitigate their debt load.
This practice ran contrary to nation-wide Republican platforms, which emphasized a politics of taxation I will detail now.

\hypertarget{the-centering-of-tax-policy}{%
\subsection{The Centering of Tax Policy}\label{the-centering-of-tax-policy}}

It was not that the NSP completely contradicted the Grand Old Party, but rather that policies affecting housing debt (and wealth) were outside the scope of national rhetoric; debt was illegible to Republican voters.
Instead, Republicans cast housing debt---and programs that help homeowners pay down housing debt---as a burden to the tax base.
But why, and how, was a question of debt transfigured into one of taxation?
Answering this question offers more than colorful background; understanding how tax and debt politics fall contingently into particular parties (or out of the mainstream altogether) bridges the gap between rational choice preferences and Congressional representative voting outcomes.
My answer also connects taxes with expenditures, a non-trivial connection that will become very important when individual preferences for social insurance are considered in the next chapter.

I argue that the tax ``revolts'' kicked off by Proposition 13 in California brought to the fore the question of taxes.
Reagan's presidential campaign then finalized the centering of taxation in Republican domestic policy.
By centering taxation---and succeeding politically in implementing favored tax cuts---the Republican Party led voters to expect tax cuts and trained taxpayers to see themselves as a political class.
With taxation in the center, and Democrats playing a softer version of Republicans' tune, debt politics became illegible as politics, rendering any small resurgence into a social movement.

Despite some notable tax activism (see: Boston Tea Party), Americans were generally quiet about taxes well into the 20th century.
This phenomenon was no doubt largely because the federal income tax fell on no one until 1913, and less than six percent of the United States population until 1940.\footnote{``Historical Sources of Income and Tax Items.''}
But after that point, the direction of tax expenditures---towards national defense in the beginning of the Cold War and for New Deal welfare programs such as the Social Security Act---were popular enough to discourage politicians from taking aim at tax cuts.\footnote{Campbell, ``How Americans Think About Taxes,'' 160.}
This equilibrium continued until the 1960s.
The fiscally-conservative Southern Democrats who controlled Congress' tax-writing committees\footnote{Campbell, 162.} were then voted out of office or changed party allegiances following federal enforcement of Civil Rights Movement legal judgments (such as \emph{Brown v. Board of Education}), the passage of the Civil Rights and Voting acts, and finally Nixon's Southern Strategy in the 1970s.

Then, in 1978, Californians voted in favor of Proposition 13, which limited the property tax rate to 1\%.
Proposition 13 emerged out of a long-fought battle by tax activists against the rationalization and modernization of property tax assessment.\footnote{Martin, ``Welcome to the Tax Cutting Party.''}
Tax assessors declined to re-appraise property values at common intervals, leading to ``fractional assessment'' that, in 1971, ``was ten times greater than the home mortgage interest deduction,''\footnote{Martin, 9.} mentioned above as a quintessentially bouregois example of the relatively-regressive American tax policies.
The success of Proposition 13 realized the potential of anti-tax movements in post-war, post--New Deal American politics.
Interviews of Congressional staffers at the time referred to a `Proposition 13 mentality'\footnote{Kingdon, \emph{Agendas, Alternatives, and Public Policies}, 97.} and President Jimmy Carter said the result ``sent a shock wave through the consciousness of every public servant.''\footnote{Jarvis and Pack, \emph{I'm Mad as Hell}, 3.}
Interestingly, Proposition 13 was not destined for the Republican platform from the outset.
Isaac William Martin refutes this perception by noting the colorful array of leftists, hippies, anti-government organizers, and 9-to-5-ers that marched in the streets in support of the referendum, while President Carter's comment above highlights the immediacy with which Democrats understood Proposition 13's political effects.
Rather, it was Reagan's signature income tax cuts that assimilated the anti-tax fervor to the Republican platform.

It is important to note now how my argument has developed with respect to a theory of political change.
In Chapter \ref{actors-motive}, I briefly engaged farmer activism in the Great Depression, a mass movement that saw thousands pack courthouses and politicians' offices.
Here, I make a slight turn: while again activism prompted radical shifts in policy, the Republican turn did not feature near-intimidation tactics, nor did ``an astonishing array of antediluvian automobiles {[}\ldots{} swarm{]} over the capitol.''\footnote{Fliter and Hoff, \emph{Fighting Foreclosure}.}
The tax revolt ``did not change what most people thought about taxes, {[}but{]} it \emph{did} change how much the major parties paid attention to taxes.''\footnote{Martin, ``Welcome to the Tax Cutting Party,'' 127. Emphasis Martin.}
This transformation moves closer to accounting for political change in the choices of elite actors than my history of the mortgage moratoria, but it still finds its roots in grass.
The next movements are much different; they see political strategists and small, connected groups of activists introduce citizens to ever more radical suggestions.
I understand this shift as taking place in a moment of transition: when there suddenly opens unsettled ground to be claimed, it is the elites who decide what flag it will fly.
The ground was unsettled in the sense that the partisan coding of anti-tax policies was ambiguous---the two parties had converged more or less---where Democrats had backed debt politics before the mortgage moratoria.

In the final centering of tax policy on the national level, Ronald Reagan would be the agent of change.
The explosive success of Proposition 13 changed Reagan's mind on taxes.
Before, he had been convinced that tax limits would just shift the burden of taxation onto one of its other forms in personal or corporate taxation.\footnote{Martin, 129.}
This belief came from a connection made between revenue and expenditure, a logic of reaping what one sows.
In California, however, local governments did not collapse, and what changes did occur to the provision of social services evidently did not summon a pro-tax backlash; in other words, ``you could cut taxes deeply without worrying about deficits.''\footnote{Martin, 130.}
The reality about this de-coupling is very important to my argument.
There have \emph{always} been large chunks of the federal budget that could legally be trimmed.
Yet, the focus of balancing the deficit rarely falls (and rarely fell) on this most massive discretionary chunk, military spending.
The purported want to balance budgets---echoed by every recent president---does not imply a need to contain visible (or invisible) welfare programs and transfers to local governments.
Rather, it is by political construction that programs like the Neighborhood Stabilization Program end up in budget debates.

Reagan's 1980 presidential campaign picked up on the Proposition 13 rhetoric, mentioning taxes in more times than any previous candidate had in his nomination acceptance speech.\footnote{Campbell, ``What Americans Think of Taxes.''}
He offered tax cuts within a larger theory, connecting the private economic conditions with larger social problems.
These would respond, not to the popular New Deal provisions, but to the increases in expenditure caused by Lyndon Johnson's Great Society, not yet 20 years old.
Taxes were certainly not the only issue on his platform, but they are that for which he is remembered: the Iranian hostage situation and antidote to Jimmy Carter's ``Crisis of Confidence'' sermon were contingencies where taxes were a certainty.
And his signature tax cut---the Economic Recovery Tax Act of 1981 that spurred savings and loan banks to sell off mortgages---led to a huge contemporary and lingering spike in popularity.(REFERENCE NEEDED)

After the reforms of the first term, the campaign ramped up its tax rhetoric.
Figure \ref{fig:taxation} shows the massive spike in tax mentions in nomination acceptance speeches (\ref{fig:acceptance}) and general election television ads (\ref{fig:ads}) aligning with the 1984 election.
Bolstered by the knowledge that one could uncouple taxes from spending, Reagan was able to fend off Democratic challenger Walter Mondale's criticism that, ``Reagan will raise taxes and so will I. He won't tell you.''\footnote{Raines, ``Party Nominates Rep. Ferraro; Mondale, in Acceptance, Vows Fair Policies and Deficit Cut.''}
Ronald Reagan would go on to win every state except Minnesota, Mondale's home.
\begin{figure}

{\centering \subfloat[Number of tax mentions by party in presidential nomination acceptance speeches.\label{fig:acceptance}\label{fig:taxation1}]{\includegraphics[width=0.45\linewidth]{figure/campbell} }\subfloat[Percentage of general election TV ads to feature taxation.\label{fig:ads}\label{fig:taxation2}]{\includegraphics[width=0.45\linewidth]{figure/campbell2} }

}

\caption{Taxation mentions by presidential campaigns.}\label{fig:taxation}
\end{figure}
After Ronald Reagan, it was the House Republicans who fueled the drive towards lower taxes.
Isaac William Martin takes over here:
\begin{quote}
Gingrich derided {[}balanced-budget Republicans'{]} argument for tax increases as an ``automatic, old-time Republican answer.'' He thereby implied that the \emph{new} Republican answer was to cut taxes even when there was a deficit. After President Reagan signed tax increases in 1982 and 1983, Gingrich was instrumental in rewriting the Republican Party platform in 1984 to repudiate further tax increases. In 1990, he broke ranks with President George H. W. Bush and led House Republicans in voting down the president's budget---because it included a tax increase.\footnote{Martin, ``Welcome to the Tax Cutting Party,'' 133.}
\end{quote}
I dwelt on Ronald Reagan's tenure because it was an inflection point in national politics.
The breakdown of Fordism in the United States, and its replacement by asset-backed political economy, was aided by Reagan's policies.
In contrast, the crystallization of taxation as dominant Republican issue, while crucial to telling the story of how past became present, is less important to understanding how housing and taxation function conceptually in American politics.

As such, I will race through Clinton's presidency with equal pace.
Notable in the Campbell graphs are the mentions of taxation by Democrats in the Clinton era.
So, while Gingrich drove legislation such as the Contract with America and the Personal Responsibility and Work Opportunity Act through Congress, it was not he---nor Republicans---alone.
In addition, Clinton co-opted what had been Republican positions on welfare, and his labeling as a Third Way or New Democrat can be seen as trying to distance his politics from those of the party generally.
Clinton's focus on budget balance differed from what Reagan and Gingrich had pushed, re-coupling revenue with expenditure, but the expenditure in focus was (as always) welfare.
By ``end{[}ing{]} welfare as we know it,'' Clinton coupled preference for social insurance with preference for higher taxes.
After Clinton, the presidential campaigns of Barack Obama and John Kerry would return Democratic campaigns to emphasizing the need for higher taxes and more supportive (visible) welfare programs.

Following Isaac William Martin's argument about the importance of Proposition 13, I have argued that tax activism propelled Republican victories in the 1980s.
Successive legislative and electoral efforts would then de-couple taxes from government spending, leaving a narrow set of discretionary spending over which candidates and legislators could argue.
This set emphasized visible welfare programs (such as Temporary Assistance for Needy Families) that socialized insurance across the tax base.
By centering taxes in the national conversation, individual debt politics became increasingly illegible to voters.
Additionally, the connection between welfare spending and taxes meant that those who felt secure financially would opt to keep otherwise-taxed income instead of the social insurance policies those funds were supposedly directed towards.
The connection between spending and taxation, however, fades from view somewhat in the early 21st century, as massive funds are appropriated for the War on Terror while Bush's first tax cuts remain deep.
Reconnecting those threads were the work of anti-tax organizations and their allies in Congress, with Obama's emergency stimulus packages offering the opportunity of a political career.

\hypertarget{emergence-of-the-tea-party}{%
\subsection{Emergence of the Tea Party}\label{emergence-of-the-tea-party}}

The Tea Party rejuvenated the connection between spending and taxes in response to President Barack Obama's stimulus bill.
The American Reinvestment and Recovery Act of 2009 (ARRA) injected \$787 billion into the United States economy via direct federal spending and indirect transfers to states and localities, {[}re-{]}authorizing a range of programs including the Neighborhood Stabilization Program.
Until then, debate about the national debt had fallen slack as the War on Terror justified huge amounts of spending to Republicans (and Democrats, at first), while Democrats since the New Deal have accommodated larger deficits in return for social insurance programs.
With a Democrat in the White House, anti-tax organizers pointed to the increase in welfare spending as an unconsented-to burden on the American taxpayer and steps down the road to federal bankruptcy.
This movement came to knock on the doors of troubled American homeowners when Rick Santelli, a reporter for CNBC, launched into an tirade from the trading floor of the Chicago Mercantile Exchange at the thought of the American taxpayer ``subsidizing the losers' mortgages,'' calling for a ``Chicago Tea Party in July.''\footnote{``CNBC's Rick Santelli's Chicago Tea Party.''}

The rant transfigured debt relief into a taxation issue, bundling it with ARRA's other increases, and making debt relief suddenly legible to the Republican party.
In the following days, prominent conservative and anti-tax organizations, such as the Heritage Foundation and Americans for Tax Relief, convened conference calls to capitalize on the opportunity.
Rallies around the country and demonstrations by smaller groups of organizers were backed by anti-tax organizations and wealthy conservatives such as the Koch brothers.
While this movement garnered criticism that the facially grassroots activity was mere ``astroturfing'', it may have nonetheless propelled Republicans to win back the House of Representatives in the 2010 midterm elections.
The Tea Party, then both formal Congressional caucus and loose alliance of conservatives, sharpened the connection between government spending and taxation that had been dulled by Bush's first term.
Their politics of debt relief substituted the national debt for household debt, and transfigured the latter into an opponent of anti-tax sentiment.

Figure \ref{fig:publicdebt} shows the total federal public debt as a percentage of GDP.
With the exception of Bill Clinton's second term from 1997 to 2001, it grew at various rates since the Economic Recovery Tax Act of 1981.
And until Obama's stimulus bill, debt politics entered the national conversation exclusively as \emph{the} national debt.
The debate about the debt shaped the presidential race most dramatically: a nationally-televised town hall debate between Clinton, George H. W. Bush, and Ross Perot in 1992.
An ill-informed question about the effects of the national debt on the candidates' lives is cited by campaign advisers and political analysts as an inflection point in the race when Clinton could connect and Bush could not.(REFERENCE NEEDED)
This example serves only to highlight the importance of the question in 1992, and more broadly in the late 1990s and early 2000s, when concerns about America's economic power were heightened by the growth of Japanese and Chinese foreign direct investment.
The volume of rhetoric surrounding the debt was perhaps a reason that bipartisan spending cuts were able to pass both houses of Congress.
\begin{figure}

{\centering \includegraphics[width=0.9\linewidth]{figure/publicdebt} 

}

\caption{Federal public debt as a percentage of gross domestic product.}\label{fig:publicdebt}
\end{figure}
Outside the Clinton presidency, however, debt became the norm.
Despite H. W. Bush's promise of ``No new taxes,'' Democrats passed the Omnibus Budget Reconciliation Act of 1990, which raised taxes and introduced restrictions on expenditures in an effort to combat deficit spending, but was unsuccessful in slowing the growth of debt relative to gross domestic product.
A generation later, George W. Bush would trigger a very moderate rise in debt-to-GDP, but a very large rise in real (2017) dollars, by increasing military spending from \$433 billion in 2001 to \$707 billion in 2008\footnote{``Military Expenditure by Country, in Constant (2017) US\$ M., 1988-2018.''} and slashing taxes.
Finally, the combination of Bush-era bailouts and Obama-era measures to combat the Global Financial Crisis and Great Recession account for the shark fin--looking feature towards the end of the 2000s.

While public opinion polling tracks the spikes and dips of this graph reasonably well, it is confounded by the War on Terror, an expenditure (initially) justifiable by members on both sides of the aisle.
In 2002, their first poll since the attacks of September 11, 2001, Pew Research Center tracked a plunge in the percentage of respondents who say that ``reducing the budget deficit is a top priority.''\footnote{``Budget Deficit Slips as Public Priority.''}
Figure \ref{fig:deficit} shows this trend by political party affiliation.
Republican-affiliated respondents in 2002, who by the turn of the millennium differ little in number from Democrats and Independents, distance themselves from both affiliations, registering the smallest number of respondents who prioritize the budget deficit between 1994 and 2016.
Priorities of the two parties reconvene in 2009, when the economic woes of Bush are inherited by Obama (Pew conducted the poll in mid-January), but diverge once again in 2011, as the new Congress is sworn in following 2010's midterm elections.
\begin{figure}

{\centering \includegraphics[width=0.9\linewidth]{figure/deficit} 

}

\caption{Percentage of respondents who say that reducing the budget deifict is a top priority by party affiliation.}\label{fig:deficit}
\end{figure}
While increasing Republican rhetoric about balanced budgets and taxation has correlated nicely with Democratic presidencies (and the reverse for Democratic rhetoric) since at least the Nixon administration,\footnote{Campbell, ``What Americans Think of Taxes.''} the massive expenditure demanded of the government (see Chapter \ref{actors-motive}) compounded the Republican case that Obama-era spending constitute profligate spending.
For Republican critics, the target of choice against Democratic presidents is simple: visible welfare spending.
Where Clinton neutralized this criticism by signing into law welfare reform, Obama touted programs during the Recession.
Organizations like the Heritage Foundation latched on to this narrative in a 2010 report entitled ``Confronting the Unsustainable Growth of Welfare Entitlements'':
\begin{quote}
According to Obama's published budget plans, means-tested welfare spending over the next decade will total \$10.3 trillion, not including spending for Obamacare. Most of this welfare spendathon will be financed by borrowing from future generations. Not surprisingly, the federal debt will grow to equal nearly the entire national economy by the end of the decade. ¶ This endless spending growth is unsustainable and will drive the nation into bankruptcy.\footnote{Bradley and Rector, ``Confronting the Unsustainable Growth of Welfare Entitlements.''}
\end{quote}
This passage, like the report as a whole, argues that ``means-tested welfare''\footnote{`Means-tested' describes a provision of a public good and/or subsidy that requires of its recipients certain behavior (such as a work requirement) or a certain status, namely that wealth or income be \emph{below} a particular level. It does not, despite first impressions, provide benefits only to people of means, as do invisible welfare programs such as the mortgage interest deduction.} would be responsible for the impoverishment of the US.
It draws a connection between the `welfare spendathon' and national debt, though it does not explicitly say---nor give evidence for the argument---that means-tested welfare will cause the American economy to equal the US gross domestic product.
I assert that this frame of argument is common among anti-tax advocates, the placement of blame solely on welfare spending for the sin of unbalanced budgets and high taxes (connected elsewhere in the report).

The Heritage Foundation report is a thorough, well-paced, academic echo of the Santelli rant televised to the nation on February 19, 2009.
Santelli's speech recasts the foreclosures of millions---and the mortgage debt of millions more---as a burden on taxpayers, indistinct from unemployment insurance, Medicare, and food stamps.
Standing on the trading floor of the Chicago Mercantile Exchange, Santelli opens with the claim that ``The government is promoting bad behavior!'' seeing crises as openings through which moral hazard creeps.
In his eyes, the goal of American economic and social policy is to promote personal responsibility and combat theories of change used by countries like Cuba, where they ``moved from the individual to the collective.''
The foreclosure crisis is not about ``subsidize{[}ing{]} the losers' mortgages,'' but using market forces to weed out those who make irresponsible or unprofitable investments ``and give 'em (the foreclosed houses) to people that might have a chance to actually prosper down the road and reward people that could carry the water instead of drink the water.''
Santelli latches on to personal responsibility and strong property rights against a ``want to pay for your neighbor's mortgage that has an extra bathroom and can't pay their bills.''\footnote{``CNBC's Rick Santelli's Chicago Tea Party.''}
His speech leverages Republican enemies (Communist-controlled Cuba, irresponsible debtors, economic inefficiencies) and friends (the individual, the entrepreneur, personal wealth) to slot this unfamiliar mortgage default problem into a more familiar narrative about personal responsibility and taxpayer burdens.

This moment coincided with burgeoning anti-tax organizations around the country.
Americans for Prosperity, founded and backed by brothers David and Charles Koch, had advocated around the idea of a formal, political Tea Party since 2002.\footnote{Zuesse, ``Final Proof the Tea Party Was Founded as A Bogus AstroTurf Movement.''}
The organization built out chapters around the country, with state chapters growing in power and turning membership rolls into legislative victories.\footnote{Skocpol, Hertel-Fernandez, and Tervo, ``How the Koch Brothers Built the Most Powerful Rightwing Group You've Never Heard of.''}
After the Santelli affair, rallies of up to a few thousand protesters gathered in cities around the United States, and in September 2009, tens of thousands marched on Washington.
The movement had outsize influence due to its insurgency, receiving a jolt of energy when Dave Brat unseated House Majority Leader Eric Cantor, a first in American primary politics.\footnote{Linton, ``House Majority Leader Eric Cantor Defeated by Tea Party Challenger David Brat in Virginia GOP Primary.''}

\hypertarget{conclusion}{%
\section{Conclusion}\label{conclusion}}

The Neighborhood Stabilization Program, by insulating homeowners from the wiles of the market, sought to construct an artificial environment where ownership would bring success.
By this aim, the NSP fit in President George W. Bush's ownership society doctrine, and the asset-backed political economy that had propelled presidential campaigns since Ronald Reagan's second term as well as the American economy more generally.
However, this asset-backed political economy had not received the functional blessing of the Republican party. At the subnational level, the party focused far more on taxation than on assets; these priorities were complementary (or at least independent of each other) so long as asset prices rose.
When these prices fell, and an economy that had reconfigured itself to feed off rising asset prices sputtered, the governmental outlays required to remedy the situation totaled several hundred billion dollars.
This price tag met simmering rhetoric about balanced budgets and decreased welfare spending, neither of which came to pass under the economic crisis policies of Bush's later and Obama's early years.
Preheated by wealthy conservatives, anti-tax organizers heeded the call of Rick Santelli to advocate for lower taxes coupled with lower spending, all wrapped up in the Tea Party movement.

While this chapter aims to connect higher housing prices to lower welfare spending to lower taxes to voting for Republicans (or at least Tea Party Republicans) in 2010, the very fact of its existence speaks to the historical contingency of such a connection.
In the next chapter, I will review political psychology literature to get at a model of voter preferences.
Where it deals directly with this connection between housing prices and right-wing voting, it neglects the coincidence of interests that led to such a possibility.
I counter its ahistoricism with this chapter, which also serves as sort of a macrotheory of the response to this crisis: where the next chapter will look at individuals, I look here at the groups which shape the individuals.

\hypertarget{methods}{%
\chapter{Asset Prices, Individual Preferences, and Foreclosure Relief}\label{methods}}

\epigraph{Keynes took it for granted that current consumption expenditure is a highly dependable and stable function of current income.[@friedman1957introduction, 1]}

How do asset prices affect voter decisions \emph{in extremis}?
Political economists have investigated the impacts of wealth on welfare preferences and labor market risk on welfare preferences, but it has not investigated how such policy preferences are represented in political decision-making, nor how such preferences change along the wealth distribution.
This analysis seeks to use the Neighborhood Stabilization Program, which aimed to affect neighborhoods in greatest need, to tease out these impacts, lengthening the focal point from groups to individuals.

This chapter begins by reviewing the relevant literature on asset-based social insurance preferences, psychology of lost and gained wealth, and American voting behavior.
It works towards a model of voting to analyze returns from the 2010 midterm elections, when the Tea Party experienced its first and greatest victories.
This chapter stands logically between my last two.
It describes how the financial crisis---whose characteristics were explored in Chapter \ref{actors-motive}---mutated individual preferences, preferences which could then be located in the political landscape described in Chapter \ref{motive-opportunity}.

\hypertarget{home-prices}{%
\section{The Role of Home Prices}\label{home-prices}}

At base, I describe individual consumption preferences from a life-cycle/permanent income perspective.
Milton Friedman opened his foundational work in life-cycle analysis by noting that Keynes understood economic fluctuations and crises in terms of the past and present, as gluts or dearths of savings.\footnote{Keynes and Krugman, \emph{The General Theory of Employment, Interest, and Money}.}
What future tense that did factor into his understanding was almost immediate---the ``animal spirits'', for instance, or the plunes in the business cycle---rather than the long swings of Kuznets.
Friedman addressed the idea that individuals' income derives not simply from current income, or previous savings, but from all future earnings.
In an extreme example, a toddler able to form ``rational'' (some fuzziness on this term's definition) expectations of future income could finance their upbringing and education by borrowing aginst those predictions, this would be a rational choice for both toddler and financier.
This framework sees wealth, and therefore housing, as potentially permanent income.

But housing can also be used as a financial ``buffer stock'' against emergencies.
This refinement is motivated by historical facts about American saving habits.\footnote{Carroll, ``Buffer-Stock Saving and the Life Cycle/Permanent Income Hypothesis,'' 1.}
In 2007, about as many households saved for retirement (33.9\%) as for unexpected expenses (32.0\%).
In addition, only 56\% of households saved in 2007, with dramatically lower rates at lower levels of the income distribution.\footnote{Bucks et al., ``Changes in U.S. Family Finances from 2004 to 2007,'' 9--10.}
Regardless of whatever official statistics can show the trends in labor market risk to be, it is the perception of homeowners that such risks necessitate savings.

I argue that understanding the home doubly as an asset shapes homeowners' political decisions.
In light of their perceptions regarding risk, how one plans to pay for emergencies feeds into that person's politics.
This claim is supported in no small part by the fact that homeowners are represented worldwide by conservative political parties.\footnote{Ansell, ``The Political Economy of Ownership,'' 387.}
Of course, it could simply be the case that high-income people, those who are more likely to own a home, tend to be more conservative.
Or, perhaps those who own homes value it as ``a refuge from urban corruption,''\footnote{Stern, ``The Inviolate Home.''} and the idea of leveraging one's home instrumentally for financial gain is vulgar to such a person's traditional social values.
But these alternative narratives cannot embrace the facts of American household finances.

Understandings of housing as an asset are not foreign to homeowners.
1982, the year following Reagan's Economic Recovery Tax Act, simultaneously saw a 35-year high for personal savings and 35-year low for mortgage equity withdrawal (MEW), as a percentage of disposable income.
By 2005, these lows and highs had flipped.
Figure \ref{fig:mew} shows the outstanding inverse relationship between the ratios of personal saving and MEW to disposable income.
For at least the XX\% of Americans who use home equity to finance personal consumption, housing expands opportunities.(pretty sure I got this from your book\ldots will find later, at any rate)
\begin{figure}

{\centering \includegraphics[width=0.9\linewidth]{figure/mew} 

}

\caption{Personal saving and mortgage equity withdrawal as a percentage of disposable income.}\label{fig:mew}
\end{figure}
While these effects are not unique to housing, they are most significant in housing.
Mortgage equity exhibits greater wealth effects than investments in stocks: research by Case, Shiller, and Quigley (2005) found statistically significant increases in the consumption patterns of housing welath over stock market wealth.
Specifically, every new dollar of housing income generated 6 cents of new consumption over the same dollar in stock market income.
Economists disagree as to the cause and level of this wealth effect, but its primacy over other financial holdings is broadly agreed upon,\footnote{Lasky and Gisselquist, ``Housing Wealth and Consumer Spending.''} and fits into the view that America has transitioned into a consumption- and import-oriented economy.
Compounding the significance of housing wealth was the sheer size and volatility of housing: Ansell notes that, ``between 1985 and 2006, real house price inflation was three times greater than between 1970 and 1985, with a standard deviation almost twice as large.''\footnote{Ansell, ``The Political Economy of Ownership,'' 383.}

Subsidizing housing and home mortgages privatizes government spending otherwise recognizable as welfare.
In doing so, the policies and implementations change how citizens view the state's role in their personal finances.
More specifically, individuals' preferences for welfare depend ``on how existing policies shape their experience of individual risk.''\footnote{Gingrich and Ansell, ``Preferences in Context.''}
A stronger claim would be that policies \emph{train} beneficiaries, a view held by Isaac William Martin.
He uses this mechanism to explain the great political diversity of organizers for Proposition 13 in California: homeowners protect the welfare afforded to them.\footnote{Martin, ``Welcome to the Tax Cutting Party.''}
At the local level, homeowners have long lobbied as a class, enclosing their neighborhoods from Black applicants, breaking fervent upzoning factions, and rejecting otherwise welcome industry.
Thus, it should come as no surprise that homeowners react to direct and indirect subsidies in similar ways as they react to market-caused fluctuations in their home values.

By privatizing government spending, the subsidies offered by the Neighborhood Stabilization Program individualize housing gains, differentiating this form of ``invisible welfare'' from the bureaucratic programs debated by the Republican and Democratic parties.
Visible welfare programs transfer individual risk to social risk,\footnote{Gingrich and Ansell, ``Preferences in Context.''} and make more equitable\footnote{Prasad, \emph{The Land of Too Much}, 229.} the public goods of a society (in the Rawlsian sense).
These two features distinguish the welfare that is argued about on television from the welfare that is argued about in journals.
The political scientist's point is that they are substitutable: preference for this private form of insurance reduces the demand for social insurance programs.

Then, the task for the individual political actor is simply to follow these preferences to their political implications by stepping back through the argument in Chapter \ref{motive-opportunity}.
This argument says that, under the narrative of balanced budgets and concerns over the national debt, lower expenditure makes possible lower revenue.
Therefore, lower preference for social insurance justifies preferences for lower taxation.
Tea Party candidates espoused this logic, clearing a spot into which voter preferences could fit.

However, it is important to understand that, just like the contingencies of party platforms, there are contingencies in how individuals process economic data and political propositions.
Larry Bartels (2002) accepts this basic assumption while rejecting the idea that ideology or partisanship are static filters through which people see selectively.
Rather, he argues, partisanship is a dynamic negotiation between partisanship as exogenous or endogenous variable.
His Bayesian learning model presents a different model: people update beliefs as new information becomes available, but people also selectively intake information according to beliefs.\footnote{Bartels, ``Beyond the Running Tally.''}
Disentangling the two sub-routines is difficult, if not impossible, but the rule of thumb is that long-held beliefs require {[}relatively{]} huge inconsistencies to catalyze change.

While the mortgage crisis, financial meltdown, and ensuing Great Recession delivered onto unsuspecting citizens heaps of information---much of it likely inconsistent with individual beliefs about meritocracy, personal financial stability, and government oversight---survey data suggests that the government response delivered similarly-sized heaps of information.
Pew Research's American Values polling series has tracked opposition and support of various political positions since the late 1980s.\footnote{Kohut et al., ``Partisan Polarization Surges in Bush, Obama Years.''}
This longitudinal sureying shows sharp divergence between the economic beliefs of Democrats and Republicans \emph{as well as} between independents who lean either way.
For partisan divides, Figure \ref{fig:partisan} shows how Republican beliefs regarding social insurance regress to a conservative ideal more aggressively than both Democrats and independents.
I split hairs over this behavior with Ansell (2014), who maintains it is self-identifying liberals who maintain an ideological opposition to spending cuts.\footnote{Ansell, ``The Political Economy of Ownership,'' 387.}
This position is not exactly borne out in the Pew numbers: while independent support for raising social insurance spending (middle section) falls relative to Democratic support after 2007, independent support for maintaining (left section) or expanding (right section) social insurance spending when disconnected from budgetary imbalances hews closely to Democratic behavior.
When these are combined with the polling on ideological slant, the Pew research offers a coherent picture of independent voting behavior between 2007 and 2012.

Figure \ref{fig:ideological} shows how independents who lean towards a party compare with those parties on a number of political positions.
The larger variance in positions for both sets of ``leaners'' shows how independents less consistently assimilate perceptions of evolving economic realities into their political positions.
Those who identify as Republicans or Democrats are not swayed in their support for social safety net provisions by the eruption of financial and economic turmoil.
In contrast, leaners towards both parties are strongly affected by its event, and even more strongly affected by the subsequent increase in spending and its surrounding rhetoric.
Partisanship affects how individuals fit economic perceptions into their political positions; not all voters \emph{could} (let alone \emph{would}) be affected by something like rising home prices.
Therefore, my model aims at marginal voters, those independents whose picture of the world is more open to alteration.
\begin{figure}

{\centering \subfloat[Support for social insurance programs by party.\label{fig:partisan}\label{fig:ideological-social1}]{\includegraphics[width=0.62\linewidth]{figure/partisan_social} }\subfloat[Support for social insurance programs by ideology.\label{fig:ideological}\label{fig:ideological-social2}]{\includegraphics[width=0.31\linewidth]{figure/ideological_social} }

}

\caption{Ideological and partisan divides over social insurance programs.}\label{fig:ideological-social}
\end{figure}
And I expect the effects to be marginal, as the Neighborhood Stabilization Program really enriched no one.
If each foreclosure costs, on average, \$159,000 in decreased property values,\footnote{Immergluck, \emph{Foreclosed}, 151.} then each surrounding home value decreases by only a fraction of that number.
Conversely, if a vacant property is fixed and resold, surrounding home values would only increase by a fraction.
Additionally, the hefty oversight on the NSP targeted profiteering by local actors, preventing any get-rich-quick schemes.
The Neighborhood Stabilization Program receives high marks methodologically from its under-the-radar nature: if voters knew, and could filter, their wealth gains through an ideological sieve, it could skew results.
Rather, its small effects and lack of advertising\footnote{``Guide to Marketing and Selling NSP Homes,'' 5.} (and reciprocal press coverage) simplify an analysis that would otherwise seek to account for the media narrative surrounding such a program.

While these factors simplify my analysis and attenuate its potential for strong conclusions, my model is complicated by the way that humans value equal amounts of money lost and gained.
In behavioral economics and cognitive psychology, the endowment effect describes the premium that individuals place on ownership.
Asked to trade one object given to them for another of equivalent exchange value, people demonstrate a marked preference for the object assigned to them.
The endowment effect is tied to loss aversion, which is that people experience a dollar lost much more acutely than a dollar gained.\footnote{Kahneman, Knetsch, and Thaler, ``Anomalies.''}
In the permanent income hypothesis, this behavior is not well integrated.
Even with ``buffer shocks'' added in, the hypothesis asserts that wealth shocks will affect spending linearly.

To give a concrete example, consider a home worth \$100,000 in 2000.
If that home increased in value \$50,000, a marginal propensity to consume (MPC)---the economist's term for the extra spending per dollar---of .06 would predict that \(.06 \times \$50,000 = \$3,000\) of extra consumption would result.
However, what if that home decresed in value \$50,000?
Would spending decrease \$3,000?
It seems unlikely that, in the midst of an event labeled as a `crisis' or the `Great Recession', spending would not decrease even more dramatically, and would be reluctant to rise for fear that momentary gains would be washed away by another crashing wave of home prices.
Or, maybe contractions exhibit a linear relationship with the \emph{rate} of price changes, but a second relationship results from the endowment effect.
In this second case, consumption is a function not only of price changes, but also of net positions: a homeowner spends freely so long as their home is worth more than their original purchase price, but is averse or highly-sensitive to dips below the original price.

Angrisani et al.~(2019) demonstrated that the marginal propensity to consume changed during the recession (2008--2010) from its run-up (2005--2008).
That these dates do not align perfectly with those of the foreclosure crisis (2007--2012) is not significant; these dates contain the three phases of the Neighborhood Stabilization Program as well as the 2010 midterms, and overlap significantly with the foreclosure crisis.
Angrisani et al.~used an instrumental variables approach to estimate the MPC of non-recessionary periods to be statistically insignificant, while that of the recession to be 0.062.\footnote{Angrisani, Hurd, and Rohwedder, ``The Effect of Housing Wealth Losses on Spending in the Great Recession,'' 986.}
His finding that non-recessionary periods exhibit no change in marginal propensity to consume conforms to the rational expectations model of microeconomic behavior---given an \emph{expected} change in income, individuals will not alter consumption.
Rather, it is only during an unexepcted change in expectations (such as a recession) that consumption changes.
This understanding integrates Angrisani's research with most of the work on housing wealth effects pre- and post-crisis.\footnote{Greenspan and Kennedy, ``Sources and Uses of Equity Extracted from Homes''; Lasky and Gisselquist, ``Housing Wealth and Consumer Spending''; Mian, Sufi, and Trebbi, ``The Political Economy of the US Mortgage Default Crisis''; Schwartz, \emph{Subprime Nation}.}
Further, Angrisani et al.~is the only approach I have found that disaggregates housing wealth MPC before and after the crisis in the familiar permanent income hypothesis framework.

And, its findings are buttressed by two other pieces of literature.
The first is the body of work on endowment effects and the psychological costs of loss versus gain.
Angrisani et al.~acknowledges this by saying ``If, as documented in previous work, wealth losses are more likely to induce changes in behavior than gains, our estimates may reflect a more pronounced sensitivity of household spending to falling house values.''\footnote{Angrisani, Hurd, and Rohwedder, ``The Effect of Housing Wealth Losses on Spending in the Great Recession,'' 987.}
The second piece is the work done by Berger et al.
This paper understands data that points to large consumption effects as running contrary to the permanent income hypothesis, which predicts that wealth shocks will be amortized over an individuals' lifetime.
They then build a new model that attempts to model homeowener expectations based on realistic features of the housng market such as debt levels, collateralization, and variable home price expectations.
These features result in larger marginal propensities to consume, and larger effects of lost wealth than gained wealth.\footnote{Berger et al., ``House Prices and Consumer Spending.''}

The upshot this literature yields for my analysis is that neighborhoods not targeted by the Neighborhood Stabilization Program should see consumption decrease greater from the bust than consumption increased from the boom.
This behavior should heighten labor market risk, meaning that, in untargeted areas, lower Republican support should be present.
An additional---but less controversial---tension to consider is the relationship between change of wealth and total position of wealth.
In general, the marginal propensity of wealth decreases as individuals get wealthier.\footnote{Arrondel, Lamarche, and Savignac, ``Wealth Effects on Consumption Across the Wealth Distribution.''}
While there is some squirreliness regarding housing MPC at the very high end of the wealth distribution (second and third homes may truly be just assets, so the purpose would be extra consumption), given the size and location of such high wealth individuals, it is unlikely the Neighborhood Stabilization Program affected those individuals' houses.
This result implies that those tracts targeted by the NSP should be especially sensitive to its marginal price increases and therefore exhibit higher Republican voting returns.

\hypertarget{controls}{%
\section{Control Variables}\label{controls}}

I turn now to more general models of American congressional voting.
Little is made of the controlling variables in academic publications, but anyone who has toyed around with regression analysis knows that the exclusion or inclusion of one or two variables can make or break a hypothesis.
Often variables are included without apparent reason, leading to suspicions about the robustness of statistical inference.
It is therefore important for the statisical analyst to offer two ways for readers to inspect one's work.
The first way is to explain the inclusion---or sometimes, exclusion---of particular variables used to disaggregate the effects of contributory forces.
The second way, shown in the next chapter, is to present analyses that include or exclude particular variables; the common belief that if a significant association can be made despite the inclusion or exclusion of several variables, than the constancy of the association effectively verifies its ability to predict behavior.

As Congressional voting returns will be the object of analysis in my next chapter, it is important to isolate the effects of the Neighborhood Stabilization Program.
To do so, I opt to include several variables that also impact a Republican vote.
These variables are lumped broadly into three categories: visible identity, including race and age; economic identity, including income and unemployment; and cultural identity, including measures of urbanicity as well as religion.
This breakdown acts purely for organizational purposes as the blur between visible, economic, and cultural should make clear, but they nonetheless reflect a view that voting decisions can be reduced to observable characteristics.
As the predictive power of my model will make clear, this status is not the case: indiviudal forces cannot explain even half of the variation in voting returns.
Nonetheless, without polling microdata, the demographic features provided by the decennial census offer the best chance at understanding voter behavior.

\hypertarget{visible-identity}{%
\subsection{Visible Identity}\label{visible-identity}}

Visible identities are those which are involuntarily imposed.
They press, as Kwame Anthony Appiah, notes, as ``walls that hedge us in.''\footnote{Appiah, \emph{The Lies That Bind}, 189.}
Since many of these identities have become legally protected categories in the United States, their records are kept precisely and publicly.
The upshot of this feature is that records are easily available, but this availability may lead to an overestimation of their effects on voting, as researchers look where the light shines instead of where their object of interest may be found.
On the other hand, while partisan voting support may have counter-intuitive logic that necessitates complex research methodologies, rational explanations for identity-based support are often at hand in visible identities.

For instance, Black voters long supported the Republican Party as the party of Abraham Lincoln, Congress' first Black legislators, and as proponents of programs benefitting African Americans.
This support changed first with the New Deal, with Franklin Delano Roosevelt won 71\% of the votes cast by African Americans in 1936, though party identification by African Americans reamined even between the two parties over the next ten years.
During the Civil Rights Movement this support changed again, when votes for Lyndon B. Johnson's first elected term commanded 94\% of the Black vote.\footnote{Jackson, ``Blacks and the Democratic Party.''}
Some of the United States' pre-eminent segregationists were staunch Democrats, but federal action by Lyndon B. Johnson and subsequent campaigning by Richard Nixon effectively fractured Democratic support in the South.
In the contemporary Democratic Party, redistributionist welfare programs aligned with integration to remedy both the economic and social disparities between white and Black Americans, turning the Democratic Party towards the majority of African Americans.

Similar stories cannot be told for American Indians, Asian Americans, or ethnically Hispanic Americans, but each has their \emph{own} story with the Republican or Democratic parties.
While American Indians have comprised a small percentage of the United States voting population, a fact owing to their legal disenfranchisement and decimation by civilian and military forces, their vote swings heavily Democratic.
This behavior is not due to historical reasons---no party has sustained Indian issues at the national scale---but to reactions for the supply of social welfare programs.
On average, 48\% more reservation residents receive cash or food stamps from the federal government than do households in the general population.\footnote{``Public Assistance Income or Food Stamps/SNAP in the Past 12 Months for Households.''}
While their turnout rates and absolute population size is low, including a variable for their population could explain the voting behavior of Census Tracts where reservations are located.
I will employ a binary cut-off metric: should a Census Tract contain a large population of Native Americans (25\%, for instance, compared to the national average of 0.9\%), then that tract would be labeled as such.

The stories of Asian Americans and ethnically Hispanic Americans are perhaps the most fractured among racial political narratives.
The ``Hispanic vote'' is often pointed to as a core aspirational piece of the Democratic coalition, though Democratic candidates maintain a checkered record in attracting and serving ethnically Hispanic voters.
Nonetheless, there are compelling reasons to include a measure of Hispanic voters.
Until a certain point, an increased Hispanic population raises fears of white job loss, increasing conservative---and especially Tea Party---voting, though no consensus regarding this ``racial threat'' hypothesis has emerged.\footnote{Campbell, Wong, and Citrin, ```Racial Threat', Partisan Climate, and Direct Democracy.''}
As white voters compose between 60\% and 70\% of the national electorate, fears inflected through them lead to significant effects.
Interestingly, Republicans used housing policy as a way to attract Hispanic voters.
With more than 50\% of Hispanic people living in America identifying themselves as Catholic,\footnote{``The Shifting Religious Identity of Latinos in the United States.''} adding the promise of a home to conservative social values would further cement the 40\% of the ``Hispanic vote'' achieved by Republicans in the 2004 presidential election.\footnote{Abrajano, Michael Alvarez, and Nagler, ``The Hispanic Vote in the 2004 Presidential Election.''}
While the motives for voting Republican by Hispanics are complex, the reasons do not cancel out one another, so I will include their population percentages as a predictor.

On the other hand, Asian Americans have pushed back against the ``Model Minority'' myth to contextualize the fact that the median Asian American income earner---come their families (or they) from India, China, Vietnam, Japan, Korea, Iran, Uzbekistan, or the Philippines---made in 2016 \$3,330 more than the average white household.
But inequality among Asian Americans is worse than for any other group; Pew notes that income earners in the 90th percentile made nearly 11 times as much as those in the 10th percentile.\footnote{Kochhar and Cilluffo, ``Income Inequality in the U.S. Is Rising Most Rapidly Among Asians.''}
Across historical, religious, and cultural factors, it is arguable that Asian Americans are the most heterogeneous group of immigrants, which makes their inclusion in Ben Ansell's 2014 paper puzzling.
Given the contemporaneous heterogeneity of voting patterns by Asian Americans, and the shifting patterns of ethnicities throughout time, I find little reason for their inclusion in this model.

Lastly, age comes into play as a proxy for an invisible identity, that of retiree.
Fixed-income voters are highly sensitive to home price increases, because asset prices are the only variable piece of their budget.
Additionally, a wealth change amortized over a shorter lifespan will present as higher average income, and therefore consumption.
Rather than use the median or average age of the U.S. population, the decennial Census tracks the number of people in five-year bins.
Therefore, I use the percentage of people 60 years of age or older---401(k) plans can be cashed out tax-free after age 59 1/2---to estimate the amount of retirees (or those whose thoughts are largely on retirement).
Additionally, older Americans generally hold more conservative social views.
In my view, these forces cannot be disentangled without microdata on labor force participation, which could distinguish a retiree from an old person.

\hypertarget{economic-identity}{%
\subsection{Economic Identity}\label{economic-identity}}

To isolate labor market risk from income, I will use both the median household income as well as the \emph{change} in median household income.
The change offers a truer picture of employment becoming underemployment and resolves a problem presented by using the unemployment rate: the HUD estimates of foreclosure rate use the unemployment rate as a predictor of foreclosures.
Inclusion in the model would re-isolate unemployment from the estimated foreclosure rate, and bias the estimation of the foreclosure rate's significance on voting in an indeterminate manner.
However, the 2010 Neighborhood Stabilization Program uses only the \emph{change} in unemployment rate, leaving open the unemployment rate as a predictor.
The inclusion of this predictor captures effects contemporaneous to the election while avoiding the modeling problems described above.

Additionally, the Census Bureau tracks homeownership rate.
At first glance, the inclusion of the 2010 homeownership rate would appear to come at an inopportune time, as the very fluctuations in the homeownership rate are captured in the foreclosure rate.
But this capture is only partial, and, furthermore, the foreclosure rate tells nothing about the absolute levels of homeownership.
Homeownership is associated with conservative voting around the world, and the effects of social capital are well-researched.\footnote{Stern, ``The Dark Side of Town.''}

\hypertarget{cultural-identity}{%
\subsection{Cultural Identity}\label{cultural-identity}}

Three primary factors flavor voting at the cultural level: religion, urbanicity, and highest level of education attained.
Within these factors, which are so grouped because they are in some sense voluntarily elected by the person to which the characteristics are attached, specific levels are significantly associated with Republican voting, while other levels are a mixed bag.
Religion exemplifies this relationship.
Republican voting by evangelical Protestants and members of the Church of Latter-Day Saints (LDS, or the Mormon Church) is much more predictable than voting by members adherent to any religion.\footnote{``A Deep Dive into Party Affiliation.''}
The U.S. Religion Census reports county-level data on adherents to a huge number of denominations, and is the only sub-national source for all fifty states.
Conducted contemporaneously with the decennial census (though not connected, organizationally), I will join county-level statistics regarding the level of evangelical Protestants and members of LDS to the Census Tract level.

Additionally, the urban-rural divide has been noted as a political cleavage.
Lower-density areas provide fewer international transportation options, educational/translation resources, and less name recognition than do large cities.
The populations of many rural areas is implicitly limited by agricultural or industrial land holdings that prevent large turnover of residents.
And on the side of push factors, residents of rural areas are often more religious than their urban counterparts, find traditional social institutions more important, and are generally older than cities and suburbs.
These factors do not imply that those areas are in stasis, but rather that the rigidity of beliefs is compounded by the institutions and political economy of rural areas.
These beliefs, in long historical sweep, are generally better represented by conservative legislators currently residing in the Republican party.

Lastly, the highest level of education attained is a huge predictor of one's party affiliation, but only for white voters.
A 15-point gap persists between white voters with and without a college degree: 45\% percent of college-educated white voters cast their ballots for Republicans, while more than 60\% of those with a college degree did the same.\footnote{Harris, ``America Is Divided by Education.''}
This trait has worsened since George H. W. Bush's presidency, before which educated and uneducated voters were more or less aligned.
Due to the racial nature of this variable, I will puruse education level \emph{multiplied} by percentage of population white as a predictor.

\hypertarget{model}{%
\section{A Model of Voting}\label{model}}

The discussion above leaves me with a model of Republican voting:
\(rep_i = \beta_0 + \beta_1forq_i + \beta_2Black_i + \beta_3AmerIndian_i + \beta_4density_i + \beta_5evang_i + \beta_6lds_i + \beta_7Hispanic_i + \beta_8Hispanic^2_i + \beta_9medHHI_i + \beta_{10}ownhome_i + \beta_{11}unemchange_i + \beta_{12}old_i + \beta_{13}(somecol_i \times white_i) + \epsilon_i\)

\hypertarget{outcome}{%
\chapter{Effects of the Neighborhood Stabilization Program on Voting}\label{outcome}}

With this model in hand, I turn now to its application.
My approach is simple but intensive: connect the votes for Republican House of Representatives candidates to foreclosure rates---taken with other variables as a rough proxy of home equity---at the Census tract level.
Logically, this approach nestles within that of Ansell 2014.
That paper connected home prices to local sentiments regarding government policies, and then looked at policies after those sentiment polls were conducted.
It imputed that, if there existed sentiments favoring less social insurance, followed by policies which enacted less social insurance, then those sentiments filtered through the voting and public opinion apparatus to produce those policies.
While Ansell's assumption that there were no significant confounding factors may have been correct, his methodology was not rigorous, as it neglected the middle step of actually voting for politicians inclined towards such measures.
My approach searches for just that---votes associated with housing prices and/or foreclosure relief policies.

Data complete with every variable in my model was available for 53.7\% of the inhabited census tracts of 24 states, including 54.8\% of those census tracts where a self-described member of the Tea Party ran for election.
Most of the remaining census tracts belonged to states where it was not feasible to connect voting returns with census geography.
For instance, Michigan does not make clear the links between a voting tabulation district (VTD) as the Census Bureau defines it, and the county-precinct-ward system the state uses to publicize its results.
For 15 other states, including Illinois, elections are so devolved that a state-wide election aggregation would have taken days to collect.
A full tabulation of data coverage is available in Table \ref{tab:coverage}.

Most of the data came by way of the US Census Bureau.
Since 1990, the Census Bureau has delineated census blocks covering the entire United States, standardizing units of analysis and making trivial the aggregation of data at various granularity.
However, some key variables in this analysis come from outside the Census Bureau, from state governments, other federal agencies, and independent researchers, each of which uses different geographical units.
To understand the construction of this novel dataset, it is important to first understand the relationships between different statistical geographies.

\hypertarget{census-geography}{%
\section{Relationships between statistical geographies}\label{census-geography}}

The basic Census geography runs as follows (see Figure \ref{fig:geography} for the Census Bureau's graphical representation\footnote{Lemery, ``LibGuides.''}).
The basic unit, blocks, were originally designed for Census takers to walk, and were constrained by physical boundaries difficult or illogical to cross.
However, since the Bureau undertook to cover the \emph{entire} United States with census blocks, there are millions of census blocks with no inhabitants, covering lakes, parks, and barren landscapes; this fact will become important shortly.
Blocks are then aggregated into census tracts (my unit of analysis): tracts contain between 1,200 and 8,000 people, and are the smallest unit for which American Community Survey data---such as median household income and educational attainment---are published.
Census tracts sit inside counties and then inside states; unlike tracts and blocks, county boundaries are also legal jurisdictions and therefore rarely change.
\begin{figure}

{\centering \subfloat[To-scale breakdown of nested census geography, with street names.\label{fig:geography1}]{\includegraphics[width=0.5\linewidth]{figure/census_small_area_geography2} }\subfloat[Organizational hierarchy of different geographies.\label{fig:geography2}]{\includegraphics[width=0.45\linewidth]{figure/geo_hierarchy} }

}

\caption{Census geography relationships.}\label{fig:geography}
\end{figure}
While tracts and blocks were \emph{designed} to be semi-permanent, and to act as the basis for other, non-Census areas, in practice their shapes shift with population changes.
These shifts mean that: not all 2000 tracts and 2010 tracts are exactly the same (though many are); and non-Census areas (such as state-designed VTDs) can cut census blocks.
Since these Census Bureau does not make public the responses of individual persons or homes until several decades after, placing each record in its appropriate area is impossible.
Rather, ecological methods approximate the distribution of persons, votes, or homes among geographies.
Ecological methods assume that persons, votes, or homes are equally distributed throughout an area: ecological methods would assume, for instance, that since six percent of Americans are millionaires, then six out of every hundred persons sampled---no matter their neighborhood---would be millionaires.
Strictly speaking, ecological methods are logical fallacies of de-composition, of taking the whole for the part.
However, guided by the right principles, ecological methods can be perfectly valid statistical tools.

The first principle is that smaller geographies are better.
Take for example tract 12099005947, a nearly-optimal tract of 4,406 people in Palm Beach County.
The owner occupation rate is 97\%, and 88\% of the residents are aged 65 or older.
However, Palm Beach County entire contains tracts with as low as 4\% 65-and-up, and owner occupancy rates around 33\%.
Taking the county for the tract would poorly represent the diversity of life within Palm Beach, smoothing out the nuances that differentiate one neighborhood's politics from the next.
Zooming in on census tracts improves the accuracy of ecological methods, allowing for a better picture

The second principle is that errors should be distributed evenly.
Errors here refers to the process by which some geographies must be decomposed and recomposed in order to translate non--2010 Census geography into 2010 Census geography.
This assumption may not always hold in my data: the history of American urban planning has shown that physical barriers---the same used to bound census blocks---often bind communities in more ways than one.
That being said, the supposed problem of partisan gerrymandering should be a moot point.
Gerrymandering operates on the state level, assigning particular voting precincts to particular Congressional districts.
Since my analysis deals with already-made and locally-created voting precincts, it skirts this particular problem.
I again rely on the granularity of this data for its validity, but without a secondary source of data (such as Zillow's or Redfin's proprietary sources), the accurate distribution of Neighborhood Stabilization Program data (which was enumerated in 2000 Census geography) will remain unknown.

The last principle is that data should be extensive.
This idea is native to all statistical inquiry, with the logic being that evenly-distributed errors will sort themselves out in large enough groups.
In this regard I have succeeded, with a 1/2 sample of the relevant elections totaling 26,550 census tracts.
To be fully open to criticism, however, I will describe the data aggregation and de/recomposition methods for each variable in my analysis.

\hypertarget{data-collection-and-aggregation}{%
\section{Data collection and aggregation}\label{data-collection-and-aggregation}}

I built my data almost exclusively from official statistics and estimates.
Given that most of my sources would eventually lead back to the Census Bureau, which covers the entire United States, I knew that the factors limiting my coverage would lie outside the Census Bureau.
Collecting and translating precinct-level voting data to census tracts proved to be the most limited and time-consuming aspect of this process.
I began by downloading precinct-level voting returns from the Harvard Election Data Archive (HEDA),\footnote{Ansolabehere, Palmer, and Lee, ``Precinct-Level Election Data, 2002-2012.''} started in 2008 by political science researchers.
This database compiled voting returns from most states and reported results for state and national races in a standardized format.

These figures were then wedded to files specifying the geographic boundaries of each voting precinct within a state.
At this stage saw the complete loss of several states' data as some systems used to identify precincts at the state-level were inconsistent with Census Bureau naming conventions.
For each state, I looked for an isomorphic mapping from state-level naming to the standardized naming of the Census Bureau or Harvard Election Data Archive, and only for links where the match was clear were the data used.
Case in point was Maryland: the voting returns data clearly contained every single specified tract in Maryland, but matching algorithms could only identify a few hundred of the several thousand voting districts.
Geographic files came primarily from the Census Bureau's TIGER/Line redistricting project, the official database for all Census geography,\footnote{``TIGER/Line.''} via the Election-Geodata mapping project, an open-source repository for redistricting data,\footnote{Kelso, ``Nvkelso/Election-Geodata.''} though, as alluded to above, some states' geographic files were also compiled by HEDA.

Translating the voting precincts to Census geography was a complicated process of decomposition and recomposition.
Since Census tracts often straddle two voting precincts, distributing votes to particular tracts relies on ecological methods, essentially peeling off the part of a precinct lying in one tract and stitching it together with parts of the other precincts that also coincide with a tract.
This process generates Frankenstein tracts, where 20\% of one precinct's votes are summed with 100\% of a second precinct's votes and 3\% of a third precinct's votes, depending on how much of a precinct coincides with a tract.
Since the voting records of these tracts are purely statistical constructions, it is perfectly fine that one have fractions of votes.
Take again, for example, our tract 12099005947 in Palm Beach County Florida: it recorded 732.00 Republican votes out of 3114.00 total votes in the 2010 Congressional race.
But it's neighbor, tract 12099005949, recorded 461.85 Republican votes against 1414.25 total votes.
Figure \ref{fig:palmbeach} shows how this geometry is possible.
\begin{figure}

{\centering \subfloat[Without census tracts.\label{fig:palmbeach1}]{\includegraphics[width=0.45\linewidth]{figure/wotracts} }\subfloat[With census tracts.\label{fig:palmbeach2}]{\includegraphics[width=0.45\linewidth]{figure/withtracts} }

}

\caption{Voting precincts and census tracts in Palm Beach County, Florida.}\label{fig:palmbeach}
\end{figure}
There are, however, more than one ways to cut a census tract.
The obvious way to slice census tracts is by geography: the spatial intersection forms the basis for dividing up votes.
The better way to slice census tracts is by voting-age population: the intersection of eligible people forms the basis for dividing up votes.
This method better leverages even more granular data that the Census Bureau publishes to partially overcome ecological approximations and account for differing population densities \emph{within} tracts.
The way tracts in exurban and coastal areas are created necessitates this approach.
Ideal census tracts \emph{are} fairly homogeneous in terms of density, as it is the goal of the Census Bureau to preserve neighborhoods and actual street blocks where possible.
In contrast, tracts containing blocks where population density begins to shift can be quite irregular.
Since tract size is constrained by population, tracts not only \emph{can} subsume those zero-population blocks I mentioned earlier, but \emph{must} subsume some uninhabited blocks.

Public Law 94-171 (PL 94-171) ensures that the Congressional districting mandate of equally-populated districts be met with small-area population counts.
In addition, requirements of the Voting Rights Act pressed states to consider the racial makeup of their Congressional districts.
Since these are governmental tabulations, the population counts were required to be public.
I drew from these data tables, comprising every block in the United States, in order to divide up census tracts.
The advantage from data granularity is large: half of the 11,166,336 census blocks in 2010 were smaller than a tenth of a square mile\footnote{``FCC Form 477,'' 1.}---Charlottesville itself contains 803 census blocks, averaging 54.14 persons.\footnote{``SF1.''}

Then, to distribute portions of voting precincts to tracts, I counted how many census blocks from each particular tract lay within each precinct.
For computational reasons, the relevant statistic here was the block's spatial center rather than its nuanced boundaries.
Since blocks are so small, and voting precincts are supposed to be fashioned from census blocks (known exceptions exist in a handful of states), using the entire boundaries of a block would likely have generated errors that had more to do with the resolution of downloadable mapping files than actual, in-practice boundaries.
Summing up the voting-age population (the census does not ask about citizenship status or felonious histories) of those blocks that lay within a particular precinct, and then comparing that figure to the total voting-age population of said precinct, allocates the percentage of votes from a precinct to a tract.
This method is represented formally in Equation \eqref{eq:vtd}, where t represents a particular tract, p a precinct from the set P of precincts intersecting with that particular tract, and b a block from the set B of blocks within that particular precinct:
\begin{equation}
votes_{t} = \sum_{p=1}^{P \cap t} votes_p \cdot \frac{\sum_{b=1}^{B \cap t} VAP_b}{\sum_{b=1}^{B} VAP_b}
\label{eq:vtd}
\end{equation}
To recap, this process first decomposes precincts P into blocks B, and then recomposes precincts from tracts T each with particular allocations of votes according to their relative share of the total precinct population.

In this dataset, each tract's votes were calculated by this method, with the exception of Californian tracts.
The Redistricting Database for the State of California, hosted at the University of California Berkeley School of Law, had already performed a more precise version of this process using voting registration records.\footnote{``The Statewide Database.''}
Voter registration records allow researchers to locate each address---and thus each resident voter---precisely within tracts and precincts.
A more careful version of my analysis would implement such methods, as the time required to aggregate each state's registration records exceeded my timeframe.

A similar process of decomposition followed by recomposition was used for the Department of Housing and Urban Development's foreclosure and unemployment data,\footnote{``Neighborhood Level Foreclosure Data''; Cody, ``Neighborhood Stabilization Program Data''; ``NSP2 Data and Methodology.''} which was tabulated at the 2000 tract- and block group--level.
The Census Bureau publishes relationship files that precisely place the percentage of housing units within each tract and block group in their 2010 equivalents.
I first decomposed 2000 tracts by the percentage of housing units intersecting with a particular 2010 tract, and then recomposed those 2010 tract figures by allocating foreclosure starts and total housing units according to the percentage each 2010 tract contained.

Lastly, data from the 2010 US Religion Census is collected only at the county-level.\footnote{Grammich et al., ``U.S. Religion Census Religious Congregations and Membership Study, 2010 (County File).''}
Since tract geography follows county geography, I simply joined those county figures to each tract, so for each tract the county average is used.
All other data used were native to the 2010 tract level, without translation.
These data were primarily from the 2010 decennial census, with some additional information from the 2005-2009 American Community Survey.
American Community Survey data comes attached with margins of error, but, no other variables had these margins of error.
In the interest of time and observations, I took the Census Bureau at their word that errors were distributed evenly.

The only data natively at the 2010 tract level from outside the Census Bureau were house price indices calculated by the Federal Housing Finance Agency.
Those data were constructed from tens of millions of federally-financed home loans.
By virtue of that source, the indices are \emph{only} composed of prime mortgages.
However, two factors work in my favor here.
First, significantly more prime mortgages than subprime mortgages defaulted during the foreclosure crisis.\footnote{Ferreira and Gyourko, ``A New Look at the U.S. Foreclosure Crisis.''}
Second, the FHFA's index only considers properties with at least 100 sales (and purchases).
This feature matters because housing prices are strongly spatial dependent: nearby homes with nonconforming mortgages will likely sell for similar amounts as homes with conforming mortgages.
Still, this limitation means that some tracts with high levels of nonconforming loans were ommitted.
There are many neighborhoods, for instance, with complete homogeneity of conforming or nonconforming loans, enclaves
often connected by mortgage brokers who handled large numbers of subprime loans or specific developments.\footnote{Dayen, \emph{Chain of Title}.}
The political behavior of these developments is interesting due to their uniformity, but beside the aims of this thesis.

\hypertarget{analysis}{%
\section{Data Analysis}\label{analysis}}

My approach will be simple: I will model the full sample under the parameters specified in \ref{model}, and then compare it to models of Tea Party--candidate tracts, with the intent to whittle conclusions down to an appropriate level of generality.
In addition, I aim to persuade readers of the validity of these models, building on the case just made for why the underlying data should be accepted.

\hypertarget{linear-models}{%
\subsection{Linear models}\label{linear-models}}
\begin{table}[!htbp] \centering 
  \caption{Full-sample linear models} 
  \label{tab:fullsample} 
\small 
\begin{tabular}{@{\extracolsep{3pt}}lcc} 
\\[-1.8ex]\hline 
\hline \\[-1.8ex] 
 & \multicolumn{2}{c}{\textit{Dependent variable:}} \\ 
\cline{2-3} 
\\[-1.8ex] & \multicolumn{2}{c}{RPCT} \\ 
\\[-1.8ex] & (1) & (2)\\ 
\hline \\[-1.8ex] 
 NSP1 & 1.800$^{***}$ & 0.110 \\ 
  & & \\ 
 NSP2 & $-$1.100$^{**}$ & $-$1.500$^{***}$ \\ 
  & & \\ 
 NSP3 & $-$5.700$^{***}$ & $-$7.500$^{***}$ \\ 
  & & \\ 
 FORQ3 & 0.480$^{***}$ & 1.100$^{***}$ \\ 
  & & \\ 
 DENSITY & 0.0002$^{***}$ & 0.0003$^{***}$ \\ 
  & & \\ 
 PRICECHG & $-$0.038$^{***}$ & 0.018$^{**}$ \\ 
  & & \\ 
 VAP\_B & $-$0.180$^{***}$ & $-$0.140$^{***}$ \\ 
  & & \\ 
 EVANRATE & 0.560$^{***}$ & 0.900$^{***}$ \\ 
  & & \\ 
 LDSRATE & 2.500$^{***}$ & 0.750$^{***}$ \\ 
  & & \\ 
 VAP\_H & 0.290$^{***}$ & 0.490$^{***}$ \\ 
  & & \\ 
 VAP\_H2 & $-$0.003$^{***}$ & $-$0.005$^{***}$ \\ 
  & & \\ 
 MEDHHI & 0.130$^{***}$ & 0.130$^{***}$ \\ 
  & & \\ 
 OWNEROCC & 0.094$^{***}$ & 0.110$^{***}$ \\ 
  & & \\ 
 UNEM10 & 0.230$^{***}$ & $-$0.046 \\ 
  & & \\ 
 OLD & 0.045$^{***}$ & 0.058$^{***}$ \\ 
  & & \\ 
 WHITESOMECOL & 0.110$^{***}$ & 0.042$^{***}$ \\ 
  & & \\ 
 VAP\_W & 0.220$^{***}$ & 0.220$^{***}$ \\ 
  & & \\ 
 FORQ3:DENSITY & $-$0.00004$^{***}$ & $-$0.0001$^{***}$ \\ 
  & & \\ 
 WHITESOMECOL:VAP\_W & 0.001$^{***}$ & 0.001$^{***}$ \\ 
  & & \\ 
 Constant & $-$21.000$^{***}$ & $-$11.000$^{***}$ \\ 
  & & \\ 
\hline \\[-1.8ex] 
Observations & 26,550 & 26,550 \\ 
R$^{2}$ & 0.490 & 0.440 \\ 
Adjusted R$^{2}$ & 0.490 & 0.430 \\ 
Residual Std. Error & 15.000 & 15.000 \\ 
F Statistic & 609.000$^{***}$ & 1,079.000$^{***}$ \\ 
\hline 
\hline \\[-1.8ex] 
\textit{Note:}  & \multicolumn{2}{r}{$^{*}$p$<$0.1; $^{**}$p$<$0.05; $^{***}$p$<$0.01} \\ 
\end{tabular} 
\end{table}
Table \ref{tab:fullsample} lists the results of the full-sample linear models with and without state-level fixed effects.
Fixed effects are a sort of catch-all method to deal with unobserved characteristics particular to certain levels.
For instance, state laws governing taxation or the shape of voting precincts are not present in my dataset, so I employ state-level fixed effects to try and isolate these effects.
Mechanically, this method groups all observations belonging to each state as a single regressor.\footnote{Due to the fact that 24 states are included in this analysis, I truncated the output in light of space. Almost all states had statistically-significant associated effects.}
While fixed-effects are standard fare in econometric analysis, I include both models because the assumptions of this fixed-effects model may not be met.
Namely, it presumes that all directly-relevant and observable predictors are present; as I will show later, these data may be too limited to fully explain the spatial correlation among the voting results.

The coefficients of most every control predictor are consistent with prior research, with two caveats.
Of particular importance, this model conforms with the literature connecting homeownership to conservative voting around the world, as the owner-occupied homeownership rate is positively associated with Republican vote share.
The 2010 unemployment rate and population density have the opposite signs that were expected: in my model, as the unemployment rate climbs, the Republican voting percentage falls; conversely, as the population density increases, so to does the vote share.
That being said, these are small effects; a (massive) 10-point rise in unemployment is associated here with just a 2.3-point rise in vote share.
For population density the impacts are even smaller.
The consistency of these impacts acts as a kind of eye test for the validity of this model, with the argument being that here are simply added predictors to an already-solid model.

Of interest to this thesis are the coefficients attached to the Neighborhood Stabilization Program, 2010 foreclosure rate, and change in home prices.
Against my hypothesis that foreclosure rates---a proxy for negative equity---would be negatively associated with Republican vote share, while home-price changes would be positively linked to Republican vote share, each regressor displays the \emph{opposite} behavior.
Part of the price-change behavior seems to be absorbed by the state-level fixed effects, however.
The logic here is that there is correlation between Republican voting shares in nearby census tracts that covaries both with the particular state and with prices.
The nearness (or farness) of tracts is unobserved by this model, but may nonetheless contain important information linked to---but distinct from---price changes.
It is ambiguous what that information may be, however: it could align with financial data on equity or rates of seriously-delinquent mortgage payments, but alternatively, it could be socio-cultural, relating instead to irrational exuberance over price changes or perceptions of fortunate homeowners.

Separate from the effects of price changes and foreclosure rates, the Neighborhood Stabilization Program had a chronologically-dependent effect on Republican vote share.
Tracts targeted with the first round of grants (NSP1) were associated with a 1.75-point increase in Republican vote share, while the later rounds---NSP2 and NSP3---were associated with decreases in Republican vote shares after taking account of race, educational attainment, religious, and economic factors.
Since properties could not be purchased (let alone repaired or resold) before funds were granted, NSP3 (and to a lesser extent NSP2) tracts could not have seen financial benefit from the program.
I interpret this result as a strong evidence for the conclusion that the Neighborhood Stabilization Program, by supporting homeowner wealth, push voters rightward.
Why, though, would NSP2 and NSP3---without a robust advertising effort or partisan attachment---be associated with Democratic votes?
My theory is, again, unobserved variable bias.
NSP1 and NSP3 were targeted due to the risk of further foreclosures, associated with high-cost, high-leverage loans (most of which were subprime) and, therefore, mortgagees at the financial margins, who tend to vote more Democratic.

The effects of the Neighborhood Stabilization Program are sharpened by the presence of Tea Party politics.
Table \ref{tab:justp} lists the results with and without state-level fixed effects for different subsets of the full sample, all of which saw races between self-avowed Tea Party candidates.
While the signs of coefficients attached to the NSP rounds remain the same, their significances vary.
This variance is likely due to sample size, as the number of tracts with winning candidates contained little more than a tenth of the total observations.
Throughout these models, most control variables remained entirely stable, with the unemployment rate and the proportion of residents 65 and older respectively losing and gaining statistical significance among models.
\begin{table}[!htbp] \centering 
  \caption{Tea-Party linear models} 
  \label{tab:justp} 
\small 
\begin{tabular}{@{\extracolsep{3pt}}lcccc} 
\\[-1.8ex]\hline 
\hline \\[-1.8ex] 
 & \multicolumn{4}{c}{\textit{Dependent variable:}} \\ 
\cline{2-5} 
\\[-1.8ex] & \multicolumn{4}{c}{RPCT} \\ 
\\[-1.8ex] & (1) & (2) & (3) & (4)\\ 
\hline \\[-1.8ex] 
 NSP1 & 0.660$^{**}$ & 1.100$^{***}$ & 1.600$^{***}$ & 0.770$^{*}$ \\ 
  & & & & \\ 
 NSP2 & $-$1.300$^{**}$ & $-$1.800$^{***}$ & $-$1.300$^{*}$ & $-$0.850 \\ 
  & & & & \\ 
 NSP3 & $-$5.700$^{***}$ & $-$5.200$^{***}$ & $-$3.800$^{***}$ & $-$0.930 \\ 
  & & & & \\ 
 FORQ3 & $-$0.210$^{**}$ & 0.330$^{***}$ & $-$0.130 & $-$0.240$^{*}$ \\ 
  & & & & \\ 
 DENSITY & $-$0.0003$^{***}$ & $-$0.0002$^{***}$ & $-$0.0004$^{***}$ & 0.00001 \\ 
  & & & & \\ 
 PRICECHG & $-$0.045$^{***}$ & 0.012 & 0.035$^{***}$ & $-$0.034$^{**}$ \\ 
  & & & & \\ 
 VAP\_B & $-$0.330$^{***}$ & $-$0.250$^{***}$ & $-$0.260$^{***}$ & $-$0.340$^{***}$ \\ 
  & & & & \\ 
 EVANRATE & 0.190$^{***}$ & 0.710$^{***}$ & 0.560$^{***}$ & 0.540$^{***}$ \\ 
  & & & & \\ 
 LDSRATE & 1.800$^{***}$ & 0.900$^{***}$ & 0.810$^{***}$ & 0.320$^{***}$ \\ 
  & & & & \\ 
 VAP\_H & 0.230$^{***}$ & 0.210$^{***}$ & 0.220$^{***}$ & 0.200$^{**}$ \\ 
  & & & & \\ 
 VAP\_H2 & $-$0.003$^{***}$ & $-$0.002$^{***}$ & $-$0.002$^{***}$ & $-$0.004$^{***}$ \\ 
  & & & & \\ 
 MEDHHI & 0.150$^{***}$ & 0.084$^{***}$ & 0.082$^{***}$ & 0.120$^{***}$ \\ 
  & & & & \\ 
 OWNEROCC & 0.061$^{***}$ & 0.110$^{***}$ & 0.110$^{***}$ & 0.093$^{***}$ \\ 
  & & & & \\ 
 UNEM10 & 0.033 & $-$0.150$^{***}$ & $-$0.042 & 0.420$^{***}$ \\ 
  & & & & \\ 
 OLD & 0.055$^{***}$ & 0.004 & $-$0.004 & 0.074$^{***}$ \\ 
  & & & & \\ 
 WHITESOMECOL & 0.160$^{***}$ & 0.110$^{***}$ & 0.099$^{***}$ & 0.170$^{***}$ \\ 
  & & & & \\ 
 VAP\_W & 0.110$^{***}$ & 0.190$^{***}$ & 0.099$^{***}$ & 0.140$^{*}$ \\ 
  & & & & \\ 
 FORQ3:DENSITY & 0.00001 & $-$0.00002$^{*}$ & 0.00001 & $-$0.0001$^{***}$ \\ 
  & & & & \\ 
 WHITESOMECOL:VAP\_W & 0.001$^{***}$ & 0.001$^{***}$ & 0.001$^{***}$ & $-$0.001 \\ 
  & & & & \\ 
 Constant & $-$2.100 & $-$2.600 & 4.400$^{*}$ & 9.500 \\ 
  & & & & \\ 
\hline \\[-1.8ex] 
Observations & 8,533 & 8,533 & 5,750 & 2,783 \\ 
R$^{2}$ & 0.630 & 0.540 & 0.440 & 0.510 \\ 
Adjusted R$^{2}$ & 0.620 & 0.540 & 0.440 & 0.510 \\ 
Residual Std. Error & 10.000 & 11.000 & 11.000 & 9.800 \\ 
F Statistic & 357.000$^{***}$ & 528.000$^{***}$ & 236.000$^{***}$ & 152.000$^{***}$ \\ 
\hline 
\hline \\[-1.8ex] 
\textit{Note:}  & \multicolumn{4}{r}{$^{*}$p$<$0.1; $^{**}$p$<$0.05; $^{***}$p$<$0.01} \\ 
\end{tabular} 
\end{table}
The variables of interest were a mixed bag.
NSP1 in each model showed strong statistical basis for rejecting the null hypothesis in favor of a positive effect on Republican voting, while the second and third rounds of the Neighborhood Stabilization Program consistently showed a negative association with Republican voting shares, though with inconsistent significance.
Due to the ever-shrinking size of the dataset, I lean towards chalking this result up to an issue of size.
The price change and foreclosure rates never delivered a statistically-significant pair consistent with expectations, with little evidence that either has a determinate effect on voting.

\hypertarget{limitations}{%
\subsection{Limitations and proposed refinements}\label{limitations}}

As stated above, these models were limited by several factors.
The three areas I focus on as most important are data quality, model fitness, and spatial dependence.
My data quality concerns are ever-present in natural-experiment methodologies.
In short, individual voting characteristics, more granular price data, and a fuller spread of census tracts could have aided the conclusivity of this evidence.
Additionally, individual-level observations would have avoided the ecological assumptions this analysis rests on.

The second question, of model fitness, leaves for more ambiguous refinements.
In inferential statistics, there is less of a need for a model to fit the data comprehensively.
On that count, the attached .40-.62 R\^{}\{2\} values are not concerning.
However, analysis of residuals leaves something to be desired.
Figure \ref{fig:linearesid} shows the residuals against fitted values for the full-sample, fixed-effects model.
Ideally, residuals show no conclusive pattern, and are independent against the fitted values.
Here, there is a clear inverse relationship framed by the bounds of the Republican voting share at 0 and 100 percent.
Linear models are not built to handle bounded response variables, though I used one for the ease of computation and interpretation.
\begin{figure}

{\centering \subfloat[Residuals of full linear model.\label{fig:linearesid}\label{fig:linear1}]{\includegraphics[width=0.45\linewidth]{figure/linearesid} }\subfloat[Acutal vs. fitted quantiles of full linear model.\label{fig:linearquant}\label{fig:linear2}]{\includegraphics[width=0.45\linewidth]{figure/linearquant} }

}

\caption{Diagnostic plots of the linear models.}\label{fig:linear}
\end{figure}
In order to validate the results of these models, I also modeled Republican vote share by different distributions.
I settled on reporting results for regressions on the Beta distribution because of its flexibility and ability to approximate the distribution of Republican voting share.
The Beta distribution is a basic (but complicated) distribution used for response variables bounded between 0 and 1.
It easily incorporates changing variance and does not assume linearity, which is helpful when considering social interactions generally, and electoral data specifically.
As Figure \ref{fig:linearquant} shows, the variance of actual versus fitted values skyrockets at the upper end of my model.
\begin{figure}

{\centering \includegraphics[width=0.9\linewidth]{figure/betaresid} 

}

\caption{Residuals of beta regression.}\label{fig:betaresid}
\end{figure}
In contrast, the behavior of Figure \ref{fig:betaresid} conforms more closely to an ideal distribution, with less-pronounced upper and lower bounds.
Results of the beta regressions can be seen in Table \ref{tab:beta}.
Coefficients are altogether similar to what the linear models predict, making me more secure in my belief of their validity.

Lastly, as mentioned, I believe these models were spatially-dependent, with biased the estimation of the home price effect on voting behavior.
I had originally planned to incorporate spatial autoregressive methods to correct for this spatial dependence, but limitations inherent to my data, as well as my own facility with such approaches, quashed that opportunity.
Nonetheless, I can virtually prove its existence in my models using standard tests of spatial correlation.

I first constructed a neighbor's matrix from geographic census tract files; this matrix tells \textbf{R} which tracts share an edge.
I then ran the residuals of each model (from a positivist interpretation, these are the unobserved variables) through Moran's I and Geary's C tests.
The similar tests look for spatial dependence respectively at the global and local levels.
Since this dataset is fractured, containing large discontiguities, Geary's C test will be the more appropriate measure; both are retained because the calculations are quick and similar.
Table \ref{tab:spdep} displays the results of these tests, all of which reject the null hypothesis of no spatial dependence at the highest precision level \textbf{R} offers.
\begin{table}

\caption{\label{tab:spdep}Results of various tests for spatial dependence. In all tests, the null hypothesis is spatially-independent.}
\centering
\begin{tabular}[t]{ccccc}
\toprule
Model & Moran's I & Moran p-value & Geary's C & Geary p-value\\
\midrule
full-sample fixed-effects & 0.82 & 0 & 0.16 & 0\\
full-sample & 0.83 & 0 & 0.15 & 0\\
Tea Party fixed-effects & 0.73 & 0 & 0.24 & 0\\
Tea Party tracts & 0.77 & 0 & 0.20 & 0\\
TP candidate lost & 0.77 & 0 & 0.19 & 0\\
\addlinespace
TP candidate lost & 0.71 & 0 & 0.25 & 0\\
\bottomrule
\end{tabular}
\end{table}
\hypertarget{conclusion-1}{%
\section{Conclusion}\label{conclusion-1}}

In sum, there is strong evidence that recipients of the Neighborhood Stabilization Program's first round of grants voted more conservative in the 2010 elections, while subsequent rounds voted more liberal.
This fact can be explained by the timescale that the NSP operated on.
Other heretofore unexamined factors, such as home price level and foreclosure rate, show mixed results, though the strongest effects came as Tea Party politics stigmatized homeownership, foreclosure, and government relief programs.

\appendix

\hypertarget{data-quality}{%
\chapter{Data Quality}\label{data-quality}}
\begin{longtable}[t]{ccccc}
\caption{\label{tab:coverage}Tract coverage.}\\
\toprule
STATE & TEAPARTY & COVERED & TOTAL & COVERAGE\\
\midrule
\endfirsthead
\caption[]{\label{tab:coverage}Tract coverage. \textit{(continued)}}\\
\toprule
STATE & TEAPARTY & COVERED & TOTAL & COVERAGE\\
\midrule
\endhead
\
\endfoot
\bottomrule
\endlastfoot
Arizona & FALSE & 459 & 610 & 75.25\\
Arizona & TRUE & 516 & 910 & 56.70\\
Arkansas & FALSE & 0 & 160 & 0.00\\
Arkansas & TRUE & 0 & 526 & 0.00\\
California & FALSE & 4393 & 5325 & 82.50\\
\addlinespace
California & TRUE & 2239 & 2699 & 82.96\\
Colorado & FALSE & 885 & 1050 & 84.29\\
Colorado & TRUE & 135 & 192 & 70.31\\
Connecticut & FALSE & 0 & 675 & 0.00\\
Connecticut & TRUE & 0 & 153 & 0.00\\
\addlinespace
Delaware & FALSE & 80 & 214 & 37.38\\
Florida & FALSE & 2005 & 3544 & 56.57\\
Florida & TRUE & 389 & 638 & 60.97\\
Georgia & FALSE & 0 & 1367 & 0.00\\
Georgia & TRUE & 0 & 590 & 0.00\\
\addlinespace
Hawaii & TRUE & 101 & 315 & 32.06\\
Idaho & FALSE & 0 & 153 & 0.00\\
Idaho & TRUE & 0 & 145 & 0.00\\
Illinois & FALSE & 0 & 1825 & 0.00\\
Illinois & TRUE & 0 & 1290 & 0.00\\
\addlinespace
Indiana & FALSE & 0 & 655 & 0.00\\
Indiana & TRUE & 0 & 852 & 0.00\\
Kentucky & FALSE & 0 & 925 & 0.00\\
Kentucky & TRUE & 0 & 185 & 0.00\\
Louisiana & FALSE & 204 & 973 & 20.97\\
\addlinespace
Louisiana & TRUE & 23 & 156 & 14.74\\
Maine & FALSE & 0 & 158 & 0.00\\
Maine & TRUE & 0 & 193 & 0.00\\
Maryland & FALSE & 127 & 1057 & 12.02\\
Maryland & TRUE & 38 & 333 & 11.41\\
\addlinespace
Massachusetts & FALSE & 271 & 612 & 44.28\\
Massachusetts & TRUE & 613 & 855 & 71.70\\
Michigan & FALSE & 0 & 1445 & 0.00\\
Michigan & TRUE & 0 & 1311 & 0.00\\
Minnesota & FALSE & 429 & 1182 & 36.29\\
\addlinespace
Minnesota & TRUE & 92 & 152 & 60.53\\
Mississippi & FALSE & 123 & 483 & 25.47\\
Mississippi & TRUE & 35 & 176 & 19.89\\
Missouri & FALSE & 64 & 914 & 7.00\\
Missouri & TRUE & 2 & 477 & 0.42\\
\addlinespace
Nevada & FALSE & 99 & 414 & 23.91\\
Nevada & TRUE & 41 & 266 & 15.41\\
New Hampshire & FALSE & 0 & 147 & 0.00\\
New Hampshire & TRUE & 0 & 145 & 0.00\\
New Jersey & FALSE & 0 & 1548 & 0.00\\
\addlinespace
New Jersey & TRUE & 0 & 454 & 0.00\\
New Mexico & FALSE & 0 & 159 & 0.00\\
New Mexico & TRUE & 0 & 339 & 0.00\\
New York & FALSE & 1154 & 2969 & 38.87\\
New York & TRUE & 1152 & 1901 & 60.60\\
\addlinespace
North Carolina & FALSE & 777 & 1359 & 57.17\\
North Carolina & TRUE & 477 & 816 & 58.46\\
Ohio & FALSE & 1002 & 1626 & 61.62\\
Ohio & TRUE & 914 & 1317 & 69.40\\
Oregon & FALSE & 0 & 492 & 0.00\\
\addlinespace
Oregon & TRUE & 0 & 334 & 0.00\\
Pennsylvania & FALSE & 1560 & 2621 & 59.52\\
Pennsylvania & TRUE & 254 & 589 & 43.12\\
South Carolina & TRUE & 513 & 1090 & 47.06\\
Tennessee & FALSE & 288 & 999 & 28.83\\
\addlinespace
Tennessee & TRUE & 136 & 490 & 27.76\\
Texas & FALSE & 1989 & 4249 & 46.81\\
Texas & TRUE & 326 & 990 & 32.93\\
Utah & FALSE & 0 & 395 & 0.00\\
Utah & TRUE & 0 & 190 & 0.00\\
\addlinespace
Vermont & FALSE & 112 & 183 & 61.20\\
Virginia & FALSE & 636 & 1357 & 46.87\\
Virginia & TRUE & 274 & 529 & 51.80\\
Washington & FALSE & 569 & 1136 & 50.09\\
Washington & TRUE & 62 & 309 & 20.06\\
\addlinespace
West Virginia & FALSE & 0 & 307 & 0.00\\
West Virginia & TRUE & 0 & 177 & 0.00\\
Wisconsin & FALSE & 791 & 1045 & 75.69\\
Wisconsin & TRUE & 201 & 347 & 57.93\\
Tea Party states & FALSE & 18017 & 44333 & 40.64\\
\addlinespace
Tea Party states & TRUE & 8533 & 22431 & 38.04\\
Tea Party states with any information & FALSE & 18017 & 33922 & 53.11\\
Tea Party states with any information & TRUE & 8533 & 15547 & 54.89\\
Total coverage of Tea Party states & NA & 26550 & 66764 & 39.77\\*
\end{longtable}
\backmatter

References

\hypertarget{refs}{}
\leavevmode\hypertarget{ref-20112010b}{}%
``2010 Census Summary File 1.'' US Census Bureau, 2011.

\leavevmode\hypertarget{ref-20112010}{}%
``2010 General Election Precinct Data.'' University of California, 2011.

\leavevmode\hypertarget{ref-20122010}{}%
``2010 TIGER/Line Shapefiles.'' US Census Bureau, 2012.

\leavevmode\hypertarget{ref-abrajano2008hispanic}{}%
Abrajano, Marisa A., R. Michael Alvarez, and Jonathan Nagler. ``The Hispanic Vote in the 2004 Presidential Election: Insecurity and Moral Concerns.'' \emph{The Journal of Politics} 70, no. 2 (April 2008): 368--82. \url{https://doi.org/10.1017/S0022381608080365}.

\leavevmode\hypertarget{ref-2015deep}{}%
``A Deep Dive into Party Affiliation.'' \emph{Pew Research Center's U.S. Politics \& Policy Project}, April 2015.

\leavevmode\hypertarget{ref-alston1983farm}{}%
Alston, Lee J. ``Farm Foreclosures in the United States During the Interwar Period.'' \emph{The Journal of Economic History} 43, no. 4 (1983): 885--903.

\leavevmode\hypertarget{ref-angrisani2019effect}{}%
Angrisani, Marco, Michael Hurd, and Susann Rohwedder. ``The Effect of Housing Wealth Losses on Spending in the Great Recession.'' \emph{Economic Inquiry} 57, no. 2 (April 2019): 972--96. \url{https://doi.org/10.1111/ecin.12753}.

\leavevmode\hypertarget{ref-ansellPoliticalEconomyOwnership2014}{}%
Ansell, Ben. ``The Political Economy of Ownership: Housing Markets and the Welfare State.'' \emph{American Political Science Review} 108, no. 2 (May 2014): 383--402. \url{https://doi.org/10.1017/S0003055414000045}.

\leavevmode\hypertarget{ref-ansolabehere2014precinctlevel}{}%
Ansolabehere, Stephen, Maxwell Palmer, and Amanda Lee. ``Precinct-Level Election Data, 2002-2012.'' Harvard Dataverse, 2014.

\leavevmode\hypertarget{ref-appiah2018lies}{}%
Appiah, Kwame Anthony. \emph{The Lies That Bind: Rethinking Identity}. EBook. New York: Liveright Publishing Corporation, 2018.

\leavevmode\hypertarget{ref-arrondel2015wealth}{}%
Arrondel, Luc, Pierre Lamarche, and Frédérique Savignac. ``Wealth Effects on Consumption Across the Wealth Distribution: Empirical Evidence.'' Working Paper. Frankfurt am Main: European Central Bank, June 2015.

\leavevmode\hypertarget{ref-bartels2002running}{}%
Bartels, Larry M. ``Beyond the Running Tally: Partisan Bias in Political Perceptions.'' \emph{Political Behavior} 24, no. 2 (2002): 117--50.

\leavevmode\hypertarget{ref-beard1913economic}{}%
Beard, Charles A. \emph{An Economic Interpretation of the Constitution of the United States}. New York: The Macmillan Company, 1913.

\leavevmode\hypertarget{ref-berger2016house}{}%
Berger, David, Veronica Guerrieri, Guido Lorenzoni, and Joseph Vavra. ``House Prices and Consumer Spending.'' Working Paper, November 2016.

\leavevmode\hypertarget{ref-boardofgovernorsofthefederalreservesystemus1949mortgagec}{}%
Board of Governors of the Federal Reserve System (US). ``Mortgage Debt Outstanding, All Holders.'' \emph{FRED, Federal Reserve Bank of St. Louis}. https://fred.stlouisfed.org/series/MDOAH, October 1949.

\leavevmode\hypertarget{ref-boardofgovernorsofthefederalreservesystemus1949mortgagea}{}%
---------. ``Mortgage Debt Outstanding by Type of Property: Farm.'' \emph{FRED, Federal Reserve Bank of St. Louis}. https://fred.stlouisfed.org/series/MDOTPFP, October 1949.

\leavevmode\hypertarget{ref-boardofgovernorsofthefederalreservesystemus1949mortgageb}{}%
---------. ``Mortgage Debt Outstanding by Type of Property: Nonfarm and Nonresidential.'' \emph{FRED, Federal Reserve Bank of St. Louis}. https://fred.stlouisfed.org/series/MDOTPNNRP, October 1949.

\leavevmode\hypertarget{ref-boardofgovernorsofthefederalreservesystemus1980mortgage}{}%
---------. ``Mortgage Debt Service Payments as a Percent of Disposable Personal Income.'' \emph{FRED, Federal Reserve Bank of St. Louis}. https://fred.stlouisfed.org/series/MDSP, January 1980.

\leavevmode\hypertarget{ref-bradley2010confronting}{}%
Bradley, Katharine, and Robert Rector. ``Confronting the Unsustainable Growth of Welfare Entitlements: Principles of Reform and the Next Steps.'' Washington, DC: The Heritage Foundation, June 2010.

\leavevmode\hypertarget{ref-bucks2009changes}{}%
Bucks, Brian K, Arthur B Kennickell, Traci L Mach, and Kevin B Moore. ``Changes in U.S. Family Finances from 2004 to 2007: Evidence Form the Survey of Consumer Finances.'' \emph{Federal Reserve Bulletin}, February 2009, 1--56.

\leavevmode\hypertarget{ref-2016budget}{}%
``Budget Deficit Slips as Public Priority.'' Washington, DC: Pew Research Center, January 2016.

\leavevmode\hypertarget{ref-campbell2009how}{}%
Campbell, Andrea Louise. ``How Americans Think About Taxes: Lessons from the History of Tax Attitudes.'' \emph{Proceedings. Annual Conference on Taxation and Minutes of the Annual Meeting of the National Tax Association} 102 (2009): 157--64.

\leavevmode\hypertarget{ref-campbell2009what}{}%
---------. ``What Americans Think of Taxes.'' In \emph{The New Fiscal Sociology}, edited by Isaac William Martin, Ajay K. Mehrotra, and Monica Prasad, 48--67. Cambridge: Cambridge University Press, 2009. \url{https://doi.org/10.1017/CBO9780511627071.004}.

\leavevmode\hypertarget{ref-campbell2006racial}{}%
Campbell, Andrea Louise, Cara Wong, and Jack Citrin. ```Racial Threat', Partisan Climate, and Direct Democracy: Contextual Effects in Three California Initiatives.'' \emph{Political Behavior} 28, no. 2 (June 2006): 129. \url{https://doi.org/10.1007/s11109-006-9005-6}.

\leavevmode\hypertarget{ref-carroll1997bufferstock}{}%
Carroll, Christopher D. ``Buffer-Stock Saving and the Life Cycle/Permanent Income Hypothesis.'' \emph{The Quarterly Journal of Economics} 112, no. 1 (1997): 1--55.

\leavevmode\hypertarget{ref-chernick2017effect}{}%
Chernick, Howard, Andrew Reschovsky, and Sandra Newman. ``The Effect of the Housing Crisis on the Finances of Central Cities.'' In \emph{Housing Markets and the Fiscal Health of US Central Cities}, 18, 2017.

\leavevmode\hypertarget{ref-cheung2014homeowners}{}%
Cheung, Ron, Chris Cunningham, and Rachel Meltzer. ``Do Homeowners Associations Mitigate or Aggravate Negative Spillovers from Neighboring Homeowner Distress?'' \emph{Journal of Housing Economics}, Housing Policy in the United States, 24 (June 2014): 75--88. \url{https://doi.org/10.1016/j.jhe.2013.11.007}.

\leavevmode\hypertarget{ref-santelli2009cnbc}{}%
``CNBC's Rick Santelli's Chicago Tea Party.'' CNBC, February 2009.

\leavevmode\hypertarget{ref-cody2010neighborhood}{}%
Cody, Mirian. ``Neighborhood Stabilization Program Data.'' Research Portal. \emph{HUD User}. https://www.huduser.gov/portal/datasets/NSP.html, October 2010.

\leavevmode\hypertarget{ref-collins2011state}{}%
Collins, J. Michael, Ken Lam, and Christopher E. Herbert. ``State Mortgage Foreclosure Policies and Lender Interventions: Impacts on Borrower Behavior in Default.'' \emph{Journal of Policy Analysis and Management} 30, no. 2 (2011): 216--32. \url{https://doi.org/10.1002/pam.20559}.

\leavevmode\hypertarget{ref-dayenChainTitleHow2016}{}%
Dayen, David. \emph{Chain of Title: How Three Ordinary Americans Uncovered Wall Street's Great Foreclosure Fraud}. New York: The New Press, 2016.

\leavevmode\hypertarget{ref-defusco2018role}{}%
DeFusco, Anthony, Wenjie Ding, Fernando Ferreira, and Joseph Gyourko. ``The Role of Price Spillovers in the American Housing Boom.'' \emph{Journal of Urban Economics} 108 (November 2018): 72--84. \url{https://doi.org/10.1016/j.jue.2018.10.001}.

\leavevmode\hypertarget{ref-dennis2011falling}{}%
Dennis, Brady. ``Falling Home Values Mean Budget Crunches for Cities.'' \emph{Washington Post}, n.d.

\leavevmode\hypertarget{ref-epstein1984revitalization}{}%
Epstein, Richard. ``Toward a Revitalization of the Contract Clause.'' \emph{University of Chicago Law Review} 51, no. 3 (June 1984).

\leavevmode\hypertarget{ref-2004fact}{}%
``Fact Sheet: America's Ownership Society: Expanding Opportunities.'' Archive. \emph{The White House: President George W. Bush}. https://georgewbush-whitehouse.archives.gov/news/releases/2004/08/20040809-9.html, August 2004.

\leavevmode\hypertarget{ref-2015fcc}{}%
``FCC Form 477: More About Census Blocks.'' Federal Communications Commission, March 2015.

\leavevmode\hypertarget{ref-ferreira2015new}{}%
Ferreira, Fernando, and Joseph Gyourko. ``A New Look at the U.S. Foreclosure Crisis: Panel Data Evidence of Prime and Subprime Borrowers from 1997 to 2012.'' Working Paper. National Bureau of Economic Research, June 2015. \url{https://doi.org/10.3386/w21261}.

\leavevmode\hypertarget{ref-fliter2012fighting}{}%
Fliter, John A., and Derek S. Hoff. \emph{Fighting Foreclosure: The Blaisdell Case, the Contract Clause, and the Great Depression}. Landmark Law Cases \& American Society. Lawrence, Kansas: University Press of Kansas, 2012.

\leavevmode\hypertarget{ref-ghent2012historical}{}%
Ghent, Andra. ``The Historical Origins of America's Mortgage Laws.'' Special Report. Research Institute for Housing America, October 2012.

\leavevmode\hypertarget{ref-gingrich2012preferences}{}%
Gingrich, Jane, and Ben Ansell. ``Preferences in Context: Micro Preferences, Macro Contexts, and the Demand for Social Policy.'' \emph{Comparative Political Studies} 45, no. 12 (December 2012): 1624--54. \url{https://doi.org/10.1177/0010414012463904}.

\leavevmode\hypertarget{ref-goodman2009mortgage}{}%
Goodman, Peter S. ``For Mortgage Servicers, an Incentive Not to Help Homeowners.'' \emph{The New York Times}, July 2009.

\leavevmode\hypertarget{ref-grammich2018religion}{}%
Grammich, Clifford, Kirk Hadaway, Richard Houseal, Dale Jones, Alexai Krindatch, Richie Stanley, and Richard Taylor. ``U.S. Religion Census Religious Congregations and Membership Study, 2010 (County File).'' Open Science Framework, 2018.

\leavevmode\hypertarget{ref-greenspan2007sources}{}%
Greenspan, Alan, and James Kennedy. ``Sources and Uses of Equity Extracted from Homes.'' Working Paper. Washington, DC: Federal Reserve Board, March 2007.

\leavevmode\hypertarget{ref-gross2012local}{}%
Gross, Liz, Kil Huh, Abigail Sylvester, and Robert Zahradnik. ``The Local Squeeze: Falling Revenues and Growing Demand for Services Challenge Cities, Counties, and School Districts.'' Edited by Susan Urahn. Philadelphia: Pew Research Center, 2012.

\leavevmode\hypertarget{ref-2010guide}{}%
``Guide to Marketing and Selling NSP Homes.'' Department of Housing and Urban Development, September 2010.

\leavevmode\hypertarget{ref-harris2018america}{}%
Harris, Adam. ``America Is Divided by Education.'' \emph{The Atlantic}. https://www.theatlantic.com/education/archive/2018/11/education-gap-explains-american-politics/575113/, November 2018.

\leavevmode\hypertarget{ref-2019historical}{}%
``Historical Sources of Income and Tax Items.'' \emph{Tax Policy Center}. https://www.taxpolicycenter.org/statistics/historical-sources-income-and-tax-items, October 2019.

\leavevmode\hypertarget{ref-1934home}{}%
``Home Building \& Loan Assn. V. Blaisdell,'' January 1934.

\leavevmode\hypertarget{ref-2002homeownership}{}%
``Homeownership Policy Book - Chapter 1.'' Archive. \emph{The White House: President George W. Bush}. https://georgewbush-whitehouse.archives.gov/infocus/homeownership/homeownership-policy-book-ch1.html, 2002.

\leavevmode\hypertarget{ref-immergluck2011foreclosed}{}%
Immergluck, Daniel. \emph{Foreclosed: High-Risk Lending, Deregulation, and the Undermining of America's Mortgage Market}. Ithaca: Cornell University Press, 2011.

\leavevmode\hypertarget{ref-immergluckPreventingNextMortgage2015}{}%
---------. \emph{Preventing the Next Mortgage Crisis: The Meltdown, the Federal Response, and the Future of Housing in America}. Lanham: Rowman \& Littlefield Education, 2015.

\leavevmode\hypertarget{ref-indiviglio2010housing}{}%
Indiviglio, Daniel. ``The Housing Stabilization Program You Haven't Heard About.'' \emph{The Atlantic}. https://www.theatlantic.com/business/archive/2010/04/the-housing-stabilization-program-you-havent-heard-about/38739/, April 2010.

\leavevmode\hypertarget{ref-jackson2008blacks}{}%
Jackson, Brooks. ``Blacks and the Democratic Party.'' \emph{FactCheck.org}, April 2008.

\leavevmode\hypertarget{ref-jarvis1979mad}{}%
Jarvis, Howard, and Robert Pack. \emph{I'm Mad as Hell: The Exclusive Story of the Tax Revolt and Its Leader}. Times Books, 1979.

\leavevmode\hypertarget{ref-kahneman1991anomalies}{}%
Kahneman, Daniel, Jack L. Knetsch, and Richard H. Thaler. ``Anomalies: The Endowment Effect, Loss Aversion, and Status Quo Bias.'' \emph{Journal of Economic Perspectives} 5, no. 1 (March 1991): 193--206. \url{https://doi.org/10.1257/jep.5.1.193}.

\leavevmode\hypertarget{ref-kelso2020nvkelso}{}%
Kelso, Nathaniel V. ``Nvkelso/Election-Geodata,'' March 2020.

\leavevmode\hypertarget{ref-keynes2007general}{}%
Keynes, John Maynard, and Paul R. Krugman. \emph{The General Theory of Employment, Interest, and Money}. New York: Palgrave Macmillan, 2007.

\leavevmode\hypertarget{ref-kingdon1995agendas}{}%
Kingdon, John W. \emph{Agendas, Alternatives, and Public Policies}. New York: Longman, 1995.

\leavevmode\hypertarget{ref-kioko2012impact}{}%
Kioko, Sharon N., and Christine R. Martell. ``Impact of State-Level Tax and Expenditure Limits (TELs) on Government Revenues and Aid to Local Governments.'' \emph{Public Finance Review}, May 2012. \url{https://doi.org/10.1177/1091142112438460}.

\leavevmode\hypertarget{ref-kochhar2018income}{}%
Kochhar, Rakesh, and Anthony Cilluffo. ``Income Inequality in the U.S. Is Rising Most Rapidly Among Asians.'' \emph{Pew Research Center's Social \& Demographic Trends Project}, July 2018.

\leavevmode\hypertarget{ref-kohut2012partisan}{}%
Kohut, Andrew, Carroll Doherty, Michael Dimock, and Scott Keeter. ``Partisan Polarization Surges in Bush, Obama Years.'' Washington, DC: Pew Research Center, June 2012.

\leavevmode\hypertarget{ref-krugman2009reagan}{}%
Krugman, Paul. ``Reagan Did It.'' \emph{The New York Times}, May 2009.

\leavevmode\hypertarget{ref-lasky2007housing}{}%
Lasky, Mark, and Andrew Gisselquist. ``Housing Wealth and Consumer Spending.'' Congressional Budget Office, January 2007.

\leavevmode\hypertarget{ref-lemery2020united}{}%
Lemery, Christopher. ``United States Census Information @ Pitt: Understanding Census Geography.'' \emph{LibGuides at University of Pittsburgh}. https://pitt.libguides.com/uscensus/understandinggeography, March 2020.

\leavevmode\hypertarget{ref-lewis2010fat}{}%
Lewis, Michael. ``The Fat Men and Their Marvelous Money Machine.'' In \emph{Liar's Poker}. W.W. Norton \& Company, 2010.

\leavevmode\hypertarget{ref-linton2014house}{}%
Linton, Eric. ``House Majority Leader Eric Cantor Defeated by Tea Party Challenger David Brat in Virginia GOP Primary.'' \emph{International Business Times}. https://www.ibtimes.com/house-majority-leader-eric-cantor-defeated-tea-party-challenger-david-brat-virginia-gop-1597736, June 2014.

\leavevmode\hypertarget{ref-lutz2011housing}{}%
Lutz, Byron, Raven Molloy, and Hui Shan. ``The Housing Crisis and State and Local Government Tax Revenue: Five Channels.'' \emph{Regional Science and Urban Economics}, Special Issue: The Effect of the Housing Crisis on State and Local Governments, 41, no. 4 (July 2011): 306--19. \url{https://doi.org/10.1016/j.regsciurbeco.2011.03.009}.

\leavevmode\hypertarget{ref-madison1788constitution}{}%
Madison, James. ``The Constitution of the United States,'' June 1788.

\leavevmode\hypertarget{ref-martin2008welcome}{}%
Martin, Isaac William. ``Welcome to the Tax Cutting Party: How the Tax Revolt Transformed Republican Politics.'' In \emph{The Permanent Tax Revolt: How the Property Tax Transformed American Politics}, 126--45. Stanford, California: Stanford University Press, 2008.

\leavevmode\hypertarget{ref-martin2015foreclosed}{}%
Martin, Isaac William, and Christopher Niedt. \emph{Foreclosed America}. Stanford, California: Stanford University Press, 2015.

\leavevmode\hypertarget{ref-mcnamara2019yale}{}%
McNamara, Christian. ``Yale Program on Financial Stability Interview,'' October 2019.

\leavevmode\hypertarget{ref-mianPoliticalEconomyUS2010}{}%
Mian, Atif, Amir Sufi, and Francesco Trebbi. ``The Political Economy of the US Mortgage Default Crisis.'' \emph{American Economic Review} 100, no. 5 (December 2010): 1967--98. \url{https://doi.org/10.1257/aer.100.5.1967}.

\leavevmode\hypertarget{ref-2019military}{}%
``Military Expenditure by Country, in Constant (2017) US\$ M., 1988-2018.'' Stockholm International Peace Research Institute, 2019.

\leavevmode\hypertarget{ref-muro2009fiscal}{}%
Muro, Mark, and Christopher Hoene. ``Fiscal Challenges Facing Cities: Implications for Recovery.'' Brookings, November 2009.

\leavevmode\hypertarget{ref-2008neighborhood}{}%
``Neighborhood Level Foreclosure Data.'' Department of Housing \& Urban Development Office of Policy Development and Research, September 2008.

\leavevmode\hypertarget{ref-comfort2010new}{}%
``New Jersey Courts Take Steps to Ensure Integrity of Residential Mortgage Foreclosure Process.'' New Jersey Courts, December 2010.

\leavevmode\hypertarget{ref-defilippo2009nsp2}{}%
``NSP2 Data and Methodology.'' Research Portal. \emph{HUD User}. https://www.huduser.gov/portal/NSP2datadesc.html, December 2009.

\leavevmode\hypertarget{ref-orolbreaks}{}%
Orol, Ronald D. ``U.S. Breaks down \$9.3 Bln Robo-Signing Settlement.'' \emph{MarketWatch}. https://www.marketwatch.com/story/us-breaks-down-93-bln-robo-signing-settlement-2013-02-28, n.d.

\leavevmode\hypertarget{ref-pelofsky2008bush}{}%
Pelofsky, Jeremy. ``Bush Signs Housing Bill as Fannie Mae Grows.'' \emph{Reuters}, July 2008.

\leavevmode\hypertarget{ref-pelosi2008housing}{}%
Pelosi, Nancy. ``Housing and Economic Recovery Act of 2008,'' July 2008.

\leavevmode\hypertarget{ref-perkins2010privatopia}{}%
Perkins, Casey. ``Privatopia in Distress: The Impact of the Foreclosure Crisis on Homeowners' Associations.'' \emph{Nevada Law Journal} 10, no. 2 (January 2010).

\leavevmode\hypertarget{ref-poterba1995stock}{}%
Poterba, James M., Andrew A. Samwick, Andrei Shleifer, and Robert J. Shiller. ``Stock Ownership Patterns, Stock Market Fluctuations, and Consumption.'' \emph{Brookings Papers on Economic Activity} 1995, no. 2 (1995): 295--372. \url{https://doi.org/10.2307/2534614}.

\leavevmode\hypertarget{ref-prasad2012land}{}%
Prasad, Monica. \emph{The Land of Too Much: American Abundance and the Paradox of Poverty}. Cambridge, Mass.: Harvard University Press, 2012.

\leavevmode\hypertarget{ref-2010public}{}%
``Public Assistance Income or Food Stamps/SNAP in the Past 12 Months for Households.'' US Census Bureau, 2010.

\leavevmode\hypertarget{ref-putnam2001bowling}{}%
Putnam, Robert D. \emph{Bowling Alone: The Collapse and Revival of American Community}. New York, NY: Simon \& Schuster, 2001.

\leavevmode\hypertarget{ref-raines1984party}{}%
Raines, Howell. ``Party Nominates Rep. Ferraro; Mondale, in Acceptance, Vows Fair Policies and Deficit Cut.'' \emph{The New York Times}, July 1984, 1.

\leavevmode\hypertarget{ref-saulny2008financial}{}%
Saulny, Susan. ``Financial Crisis Takes a Toll on Already-Squeezed Cities.'' \emph{New York Times}, June 2008, 16.

\leavevmode\hypertarget{ref-schwartz2009subprime}{}%
Schwartz, Herman M. \emph{Subprime Nation: American Power, Global Capital, and the Housing Bubble}. Cornell Studies in Money. Ithaca: Cornell University Press, 2009.

\leavevmode\hypertarget{ref-seabrooke2009politics}{}%
Seabrooke, Leonard. \emph{The Politics of Housing Booms and Busts}. Edited by Herman M. Schwartz. International Political Economy Series. Palgrave Macmillan UK, 2009. \url{https://doi.org/10.1057/9780230280441}.

\leavevmode\hypertarget{ref-skocpol2018how}{}%
Skocpol, Theda, Alexandra Hertel-Fernandez, and Caroline Tervo. ``How the Koch Brothers Built the Most Powerful Rightwing Group You've Never Heard of.'' \emph{The Guardian}, September 2018.

\leavevmode\hypertarget{ref-stein2019capital}{}%
Stein, Samuel. \emph{Capital City: Gentrification and the Real Estate State}. Jacobin Series. London ; Brooklyn, NY: Verso, 2019.

\leavevmode\hypertarget{ref-stern2013dark}{}%
Stern, Stephanie M. ``The Dark Side of Town: The Social Capital Revolution in Residential Property.'' \emph{Virginia Law Review} 99, no. 811 (April 2013): 68.

\leavevmode\hypertarget{ref-stern2010inviolate}{}%
---------. ``The Inviolate Home: Housing Exceptionalism in the Fourth Amendment.'' \emph{Cornell Law Review} 95 (May 2010): 48.

\leavevmode\hypertarget{ref-stollerHousingCrashEnd2016}{}%
Stoller, Matt. ``The Housing Crash and the End of American Citizenship.'' \emph{Fordham Urban Law Journal} 39, no. 4 (February 2016): 1183.

\leavevmode\hypertarget{ref-2014shifting}{}%
``The Shifting Religious Identity of Latinos in the United States.'' \emph{Pew Research Center's Religion \& Public Life Project}, May 2014.

\leavevmode\hypertarget{ref-tobler1970computer}{}%
Tobler, W. R. ``A Computer Movie Simulating Urban Growth in the Detroit Region.'' \emph{Economic Geography} 46 (1970): 234--40. \url{https://doi.org/10.2307/143141}.

\leavevmode\hypertarget{ref-usbureauofeconomicanalysis1929gross}{}%
U.S. Bureau of Economic Analysis. ``Gross Domestic Product.'' \emph{FRED, Federal Reserve Bank of St. Louis}. https://fred.stlouisfed.org/series/GDPA, January 1929.

\leavevmode\hypertarget{ref-2019foreclosure}{}%
``U.S. Foreclosure Activity Drops to 13-Year Low in 2018.'' \emph{ATTOM Data Solutions}. https://www.attomdata.com/news/most-recent/2018-year-end-foreclosure-market-report/, January 2019.

\leavevmode\hypertarget{ref-wheelockChangingRulesState2008}{}%
Wheelock, David C. ``Changing the Rules: State Mortgage Foreclosure Moratoria During the Great Depression.'' \emph{Federal Reserve Bank of St. Louis Review} 90, no. 6 (n.d.): 569--83.

\leavevmode\hypertarget{ref-1942wickard}{}%
``Wickard V. Filburn,'' October 1942.

\leavevmode\hypertarget{ref-zuesse2013final}{}%
Zuesse, Eric. ``Final Proof the Tea Party Was Founded as A Bogus AstroTurf Movement.'' \emph{HuffPost}. https://www.huffpost.com/entry/final-proof-the-tea-party\_b\_4136722, October 2013.


% Index?

\end{document}
